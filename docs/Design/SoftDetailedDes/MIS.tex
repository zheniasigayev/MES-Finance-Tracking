\documentclass[12pt, titlepage]{article}

\usepackage{amsmath, mathtools}

\usepackage[round]{natbib}
\usepackage{amsfonts}
\usepackage{amssymb}
\usepackage{graphicx}
\usepackage{colortbl}
\usepackage{xr}
\usepackage{hyperref}
\usepackage{longtable}
\usepackage{xfrac}
\usepackage{tabularx}
\usepackage{float}
\usepackage{siunitx}
\usepackage{booktabs}
\usepackage{multirow}
\usepackage[section]{placeins}
\usepackage{caption}
\usepackage{fullpage}

\hypersetup{
bookmarks=true,     % show bookmarks bar?
colorlinks=true,       % false: boxed links; true: colored links
linkcolor=red,          % color of internal links (change box color with linkbordercolor)
citecolor=blue,      % color of links to bibliography
filecolor=magenta,  % color of file links
urlcolor=cyan          % color of external links
}

\usepackage{array}

\externaldocument{../../SRS/SRS}

\input{../../Comments}
%% Common Parts

\newcommand{\progname}{Software Engineering} % PUT YOUR PROGRAM NAME HERE
\newcommand{\authname}{Team \#5, Money Making Mauraders
\\ Zhenia Sigayev
\\ Justin Ho
\\ Thomas Wang
\\ Michael Shi
\\ Johnny Qu
}


\usepackage{hyperref}
    \hypersetup{colorlinks=true, linkcolor=blue, citecolor=blue, filecolor=blue,
                urlcolor=blue, unicode=false}
    \urlstyle{same}
                                


\begin{document}

\title{Module Interface Specification for \progname{}}

\author{\authname}

\date{\today}

\maketitle

\pagenumbering{roman}

\section{Revision History}

\begin{tabularx}{\textwidth}{p{3cm}p{2cm}X}
\toprule {\bf Date} & {\bf Version} & {\bf Notes}\\
\midrule
Monday, November 10, 2025 & 1.0 & Initial Document\\
\bottomrule
\end{tabularx}

~\newpage

\section{Symbols, Abbreviations and Acronyms}

See SRS Documentation at \wss{give url}

\wss{Also add any additional symbols, abbreviations or acronyms}

\newpage

\tableofcontents

\newpage

\pagenumbering{arabic}

\section{Introduction}

The following document details the Module Interface Specifications (MIS) for the MES Finance Tracking Platform, a web-based system that streamlines the submission, approval, and monitoring of reimbursement requests for McMaster Engineering Society clubs. By centralizing digital expense forms, supporting documentation, and reviewer workflows, the platform shortens processing times and improves transparency for students and finance administrators alike.

This document specifies the MIS for each software module in the MES Finance Tracking Platform, outlining how components interact to deliver the functionality described in the accompanying design artifacts.

Complementary documents include the System Requirement Specifications, Module Guide, and complete documentation and implementation can be found at \url{https://github.com/zheniasigayev/MES-Finance-Tracking}.

\section{Notation}

\wss{You should describe your notation.  You can use what is below as
  a starting point.}

The structure of the MIS for modules comes from \citet{HoffmanAndStrooper1995},
with the addition that template modules have been adapted from
\cite{GhezziEtAl2003}.  The mathematical notation comes from Chapter 3 of
\citet{HoffmanAndStrooper1995}.  For instance, the symbol := is used for a
multiple assignment statement and conditional rules follow the form $(c_1
\Rightarrow r_1 | c_2 \Rightarrow r_2 | ... | c_n \Rightarrow r_n )$.

The following table summarizes the primitive data types used by \progname. 

\begin{center}
\renewcommand{\arraystretch}{1.2}
\noindent 
\begin{tabular}{l l p{7.5cm}} 
\toprule 
\textbf{Data Type} & \textbf{Notation} & \textbf{Description}\\ 
\midrule
character & char & a single symbol or digit\\
integer & $\mathbb{Z}$ & a number without a fractional component in (-$\infty$, $\infty$) \\
natural number & $\mathbb{N}$ & a number without a fractional component in [1, $\infty$) \\
real & $\mathbb{R}$ & any number in (-$\infty$, $\infty$)\\
\bottomrule
\end{tabular} 
\end{center}

\noindent
The specification of \progname \ uses some derived data types: sequences, strings, and
tuples. Sequences are lists filled with elements of the same data type. Strings
are sequences of characters. Tuples contain a list of values, potentially of
different types. In addition, \progname \ uses functions, which
are defined by the data types of their inputs and outputs. Local functions are
described by giving their type signature followed by their specification.

\section{Module Decomposition}

The following table is taken directly from the Module Guide document for this project.

\begin{table}[h!]
\centering
\begin{tabular}{p{0.3\textwidth} p{0.6\textwidth}}
\toprule
\textbf{Level 1} & \textbf{Level 2}\\
\midrule

{Hardware-Hiding} & \textbullet~M1: Hardware Hiding Module \\
\midrule

\multirow{7}{0.3\textwidth}{Behaviour-Hiding} & Input Parameters\\
& Output Format\\
& Output Verification\\
& Temperature ODEs\\
& Energy Equations\\ 
& Control Module\\
& Specification Parameters Module\\
\midrule

\multirow{3}{0.3\textwidth}{Software Decision} & {Sequence Data Structure}\\
& ODE Solver\\
& Plotting\\
\bottomrule

\end{tabular}
\caption{Module Hierarchy}
\label{TblMH}
\end{table}

\newpage
\section{MIS of Hardware Hiding Module (M1)}\label{sec:hardware-hiding}

\subsection{Module}
Hardware Hiding Module (M1)

\subsection{Uses}
None.

\subsection{Syntax}

\subsubsection{Exported Constants}
\begin{itemize}
  \item \textbf{RuntimeVersion}: string --- Declares the LTS Node.js runtime version made available by the hosting platform (Vercel/AWS) so that the Behaviour-Hiding and Software Decision modules can align their server-side features and package targets.
  \item \textbf{DefaultRegion}: string --- Identifies the primary cloud deployment region used for the platform's compute, database, and object storage endpoints to satisfy latency expectations from the SRS operational environment.
  \item \textbf{MaxConcurrentRequests}: $\mathbb{N}$ --- Upper bound on simultaneous backend invocations guaranteed by the provider, chosen to satisfy the SRS capacity requirement of supporting at least 500 concurrent users.
\end{itemize}

\subsubsection{Exported Access Programs}

\begin{center}
\begin{tabular}{p{3.5cm} p{4.5cm} p{4.5cm} p{3cm}}
\toprule
\textbf{Name} & \textbf{In} & \textbf{Out} & \textbf{Exceptions} \\
\midrule
getRuntimeContext & - & RuntimeContext & ConfigNotFound \\
provisionPersistence & PersistenceTarget, PersistencePolicy & PersistenceHandle & ProvisioningError \\
openFileChannel & StorageRequest & StorageHandle & StorageError \\
resolveSecret & SecretKey & SecretValue & SecretNotFound \\
\bottomrule
\end{tabular}
\end{center}

\subsection{Semantics}

\subsubsection{State Variables}
\begin{itemize}
  \item \textbf{activeHandles}: sequence of PersistenceHandle --- Tracks open database or object storage connections issued through the module.
  \item \textbf{cachedSecrets}: map SecretKey $\rightarrow$ SecretValue --- Maintains decrypted secrets for the lifetime of a runtime invocation to minimize calls to the provider's secret manager.
\end{itemize}

\subsubsection{Environment Variables}
\begin{itemize}
  \item \textbf{cloudPlatform}: descriptor of the underlying infrastructure (e.g., Vercel Edge Functions or AWS Lambda + S3) that supplies compute, networking, and storage abstractions.
  \item \textbf{processEnv}: map string $\rightarrow$ string exposing environment variables injected by the hosting provider, including connection strings and API keys referenced in the Development Plan.
  \item \textbf{filesystem}: ephemeral file system mount allocated per invocation for staging uploads before they are persisted to long-term storage.
  \item \textbf{networkInterface}: outbound HTTPS client managed by the runtime for communicating with third-party services such as SendGrid or future payment APIs.
\end{itemize}

\subsubsection{Assumptions}
\begin{itemize}
  \item The cloud platform guarantees availability targets and auto-scaling properties outlined in the SRS environment and capacity requirements.
  \item Required environment variables (database URIs, storage bucket identifiers, API credentials) are injected securely by the deployment pipeline defined in the Development Plan.
  \item Serverless function instances share no persistent disk state; any durable data must be written through \texttt{provisionPersistence} or \texttt{openFileChannel}.
\end{itemize}

\subsubsection{Access Routine Semantics}
\noindent\textbf{getRuntimeContext}($\,$)
\begin{itemize}
  \item transition: None.
  \item output: Returns a RuntimeContext record containing \textbf{RuntimeVersion}, \textbf{DefaultRegion}, exposed environment variables, and runtime limits (memory, execution time) advertised by \textbf{cloudPlatform}.
  \item exception: ConfigNotFound is raised if mandatory runtime metadata is missing from \textbf{processEnv}.
\end{itemize}

\noindent\textbf{provisionPersistence}(target, policy)
\begin{itemize}
  \item transition: Adds a PersistenceHandle corresponding to the requested \textit{target} (document store, object storage, or cache) to \textbf{activeHandles}. The handle encapsulates connection pooling or signed URLs as required by the target.
  \item output: Returns the newly created PersistenceHandle configured according to \textit{policy} (e.g., read/write role, retention period).
  \item exception: ProvisioningError if the target infrastructure is unreachable or policy constraints cannot be satisfied by the provider.
\end{itemize}

\noindent\textbf{openFileChannel}(request)
\begin{itemize}
  \item transition: Streams the binary payload described by \textit{request} from \textbf{filesystem} to an object storage location derived from \textbf{cloudPlatform}. Updates \textbf{activeHandles} with the resulting StorageHandle for lifecycle management.
  \item output: Returns a StorageHandle containing the canonical URI and checksum for the persisted object so that the Receipt Processing module can reference it.
  \item exception: StorageError when the payload exceeds provider-imposed limits or when the storage backend reports an error.
\end{itemize}

\noindent\textbf{resolveSecret}(key)
\begin{itemize}
  \item transition: If \textit{key} is not present in \textbf{cachedSecrets}, retrieves the secret value from the provider-managed vault and caches it for the remaining invocation lifespan.
  \item output: Returns the SecretValue associated with \textit{key} so downstream modules can authenticate with external services (e.g., MongoDB, SendGrid).
  \item exception: SecretNotFound when the requested key is absent from the vault or access is denied.
\end{itemize}

\subsubsection{Local Functions}
None.

\subsubsection{Considerations}
\begin{itemize}
  \item This module virtualizes physical infrastructure so higher-level modules can operate independent of Vercel/AWS specific APIs, satisfying the information-hiding intent of the Module Guide.
  \item Capacity thresholds published through \textbf{MaxConcurrentRequests} ensure the Request Handler and Notification modules can plan throttling logic that respects SRS capacity and scalability constraints.
  \item Secrets resolved via \texttt{resolveSecret} enable future integration with the MES monorepo while keeping credential management centralized, as described in the Development Plan.
\end{itemize}

~\newpage

\section{MIS of Receipt Processing Module (M4)} \label{ModuleReceiptProcessing}

\subsection{Module}

ReceiptProcessing

\subsection{Uses}

\begin{itemize}
  \item Authentication Module (M6) for verifying that a caller has sufficient permissions to interact with a stored receipt.
  \item Data Model Module (M7) for persisting receipt metadata, OCR extraction results, and associating receipts with reimbursement requests.
  \item Audit Logging Module (M8) for recording immutable traces of upload, access, extraction, and deletion actions.
  \item External storage and OCR services (e.g., AWS S3 and Amazon Textract) identified in the Module Guide and SRS for hosted file storage and automated data extraction.
\end{itemize}

\subsection{Syntax}

\subsubsection{Exported Constants}

\begin{itemize}
  \item \textbf{ACCEPTED\_FILE\_TYPES}: Set of MIME types $\{\texttt{image/png},\, \texttt{image/jpeg},\, \texttt{application/pdf}\}$ defining the allowable receipt formats described in the SRS.
  \item \textbf{MAX\_RECEIPT\_FILE\_SIZE}: Maximum upload size of $10$ MB, ensuring conformance with the non-functional upload latency target.
  \item \textbf{SIGNED\_URL\_TTL}: Duration (in minutes) for which a generated secure access link to a stored receipt remains valid.
\end{itemize}

\subsubsection{Exported Access Programs}

\begin{center}
\begin{tabular}{p{2.8cm} p{4.2cm} p{4.2cm} p{2.4cm}}
\toprule
\textbf{Name} & \textbf{In} & \textbf{Out} & \textbf{Exceptions} \\
\midrule
uploadReceipt & ReceiptFile, ReceiptMetadata, UserID & ReceiptRecord & InvalidFileFormat, FileTooLarge, UnauthorizedAccess, StorageFailure \\
extractReceiptData & ReceiptID & ExtractedReceiptData & ReceiptNotFound, OCRFailure, StorageFailure \\
getReceiptLink & ReceiptID, UserID & URL & ReceiptNotFound, UnauthorizedAccess, StorageFailure \\
removeReceipt & ReceiptID, UserID & $-$ & ReceiptNotFound, UnauthorizedAccess, ImmutableRequestState, StorageFailure \\
\bottomrule
\end{tabular}
\end{center}

\subsection{Semantics}

\subsubsection{State Variables}

\begin{itemize}
  \item receiptIndex: Mapping of ReceiptID to ReceiptRecord capturing storage location, uploader, request association, timestamps, and validation flags.
  \item extractionCache: Mapping of ReceiptID to ExtractedReceiptData persisted for reuse across module calls.
  \item storageCredentials: Handle with scoped credentials for interacting with the managed receipt storage service.
\end{itemize}

\subsubsection{Environment Variables}

\begin{itemize}
  \item storageService: External object storage endpoint (e.g., AWS S3 bucket) for binary receipt files.
  \item ocrService: External OCR provider (e.g., Amazon Textract) capable of parsing receipt images into structured data.
  \item auditChannel: Interface exposed by the Audit Logging Module (M8) for recording receipt lifecycle events.
\end{itemize}

\subsubsection{Assumptions}

\begin{itemize}
  \item The caller has already been authenticated and provides a UserID that maps to an existing account and role via the Authentication Module.
  \item The reimbursement request referenced within ReceiptMetadata exists and is mutable according to business rules managed by the Request Handler Module (M3).
  \item Credentials for storageService and ocrService are valid and provisioned prior to module invocation.
  \item Network connectivity to external services is available when upload, extraction, or deletion operations are attempted.
\end{itemize}

\subsubsection{Access Routine Semantics}

\noindent uploadReceipt(receiptFile, metadata, uploaderId):
\begin{itemize}
  \item transition: Validate that receiptFile MIME type is in ACCEPTED\_FILE\_TYPES and its size does not exceed MAX\_RECEIPT\_FILE\_SIZE. Verify that uploaderId is authorized to add receipts for the target reimbursement request. Persist receiptFile to storageService, create or update the associated ReceiptRecord in receiptIndex with a new ReceiptID, storage pointer, metadata, and timestamp, and emit an audit log entry.
  \item output: Return the created ReceiptRecord, including the ReceiptID assigned to the stored file.
  \item exception: Raise InvalidFileFormat if MIME validation fails, FileTooLarge if size constraints are exceeded, UnauthorizedAccess if uploaderId lacks the required role, or StorageFailure if persistence to storageService fails.
\end{itemize}

\noindent extractReceiptData(receiptId):
\begin{itemize}
  \item transition: If extractionCache already contains receiptId, reuse the cached result; otherwise, retrieve the receipt binary from storageService and invoke ocrService to parse the receipt. Persist the structured response in extractionCache and associate it with the corresponding ReceiptRecord for downstream processing and validation.
  \item output: Return the ExtractedReceiptData containing amounts, vendor, dates, and other parsed fields for the receipt.
  \item exception: Raise ReceiptNotFound if receiptId is absent from receiptIndex, OCRFailure if ocrService cannot produce a result, or StorageFailure if the receipt binary cannot be retrieved.
\end{itemize}

\noindent getReceiptLink(receiptId, requesterId):
\begin{itemize}
  \item transition: Confirm that requesterId is permitted to view the receipt according to reimbursement request visibility rules. Optionally record the access attempt through auditChannel.
  \item output: Return a time-bound, pre-signed URL (valid for SIGNED\_URL\_TTL) that allows the requester to download the stored receipt from storageService.
  \item exception: Raise ReceiptNotFound if receiptId is unknown, UnauthorizedAccess if requesterId lacks privileges, or StorageFailure if URL generation fails.
\end{itemize}

\noindent removeReceipt(receiptId, requesterId):
\begin{itemize}
  \item transition: Verify that requesterId has permission to remove the receipt and that the linked reimbursement request is still editable. Delete the receipt binary from storageService, remove the entry from receiptIndex and extractionCache, and log the removal.
  \item output: None.
  \item exception: Raise ReceiptNotFound if the ReceiptID is missing, UnauthorizedAccess if requesterId lacks rights, ImmutableRequestState if the reimbursement request is locked (e.g., already approved), or StorageFailure if deletion from storageService fails.
\end{itemize}

\subsubsection{Local Functions}

\begin{itemize}
  \item isAllowedFormat(fileMime): Boolean helper returning true when fileMime is contained in ACCEPTED\_FILE\_TYPES.
  \item isAuthorized(actorId, requestId, action): Queries the Authentication Module for the actor's roles and the Request Handler Module for the reimbursement state to ensure an operation is permitted.
  \item writeAuditEntry(event): Delegates to auditChannel to persist a structured log for compliance traceability.
\end{itemize}


~\newpage

\section{MIS of Notification Module (M5)} \label{ModuleNotification}

\subsection{Module}

NotificationModule

\subsection{Uses}

\begin{itemize}
  \item Receipt Processing Module (M4) sending a notifcation to the user upon successful upload and processing of a receipt.
  \item Authentication Module (M6) for verifying user contact details and preferences before dispatching notifications.
  \item Data Model Module (M7) for logging notification history and statuses after the messages are sent.
\end{itemize}

\subsection{Syntax}

\subsubsection{Exported Constants}

\begin{itemize}
  \item notificationTypes: Set of supported notification categories, including receiptUploaded, requestApproved, requestRejected, defining the scenarios in which users can be notified.
  \item clubId: Unique identifier for the McMaster Engineering Society club associated with the user receiving the notification.
\end{itemize}

\subsubsection{Exported Access Programs}

\begin{center}
\begin{tabular}{p{2.8cm} p{3.4cm} p{2.8cm} p{5.6cm}}
\toprule
\textbf{Name} & \textbf{In} & \textbf{Out} & \textbf{Exceptions} \\
\midrule
notifyUser & userId, notificationType, email & Email Content & InvalidEmailFormat, UnauthorizedAccess, NotificationFailure \\
\bottomrule
\end{tabular}
\end{center}

\subsection{Semantics}

\subsubsection{State Variables}

\begin{itemize}
  \item reimbursementProcessIndex: Mapping the current request process/step to the corresponding notificationType to determine when notifications should be sent.
  \item notificationIndex: Mapping of NotificationID to NotificationRecord capturing recipient, notification type, notification message, timestamp, and delivery status.
\end{itemize}

\subsubsection{Environment Variables}

\begin{itemize}
  \item None
\end{itemize}

\subsubsection{Assumptions}

\begin{itemize}
  \item There will be 2 types of notifcations. One will notify the user within the application UI that their receipt has been successfully uploaded and processed. The second will notify both the user and the club executive team when a reimbursement request has been approved or rejected via email.
\end{itemize}

\subsubsection{Access Routine Semantics}

\noindent notifyUser(userId, notificationType, notificationContent):
\begin{itemize}
  \item transition: Validate that userId is authorized to receive notifications and that notificationType is supported. Construct the notification message using notificationContent and dispatch it to the current user and the club executive team. Log the notification attempt in notificationIndex with status.
  \item output: Return the full notification content that was sent to the user.
  \item exception: NotificationFailure if the email service reports an error.  
\end{itemize}

\noindent notifyClubExecutive(clubId, notificationType, notificationContent):
\begin{itemize}
  \item transition: Retrieve the contact details of the club executive team associated with clubId. Construct the notification message using notificationContent and dispatch it to the club executive team. Log the notification attempt in notificationIndex with status.
  \item output: Return the full notification content that was sent to the club executive team.
  \item exception: NotificationFailure if the email service reports an error.  
\end{itemize}


~\newpage

\section{MIS of Data Model Module (M7)} \label{DataModelModule}

\subsection{Module}

Data Model Module

\subsection{Uses}

\begin{itemize}
  \item None, this is a foundational module that other modules depend on for data persistence.
\end{itemize}

\subsection{Syntax}

\subsubsection{Exported Constants}

\begin{itemize}
  \item databaseURL: string that represents the database connection URL
  \item maxBatchSize: value representing the maximum number of records that can be processed in a single batch operation
  \item maxConnections: value representing the maximum number of concurrent connections to the database
  \item defaultTimeout: value representing the default timeout duration for database operations
\end{itemize}

\subsubsection{Exported Access Programs}

\begin{center}
\begin{tabular}{p{2.8cm} p{3.4cm} p{2.8cm} p{5.6cm}}
\toprule
\textbf{Name} & \textbf{In} & \textbf{Out} & \textbf{Exceptions} \\
\midrule
query & queryString, parameters & queryResult & InvalidQuery, DatabaseConnectionError, TimeoutError \\
insert & tableName, receiptImage/document: Object & insertResult, recordId & InvalidRecordData, DatabaseConnectionError, TimeoutError \\
update & tableName, recordId, updatedFields & updateResult, recordId & RecordNotFound, InvalidUpdateData, DatabaseConnectionError, TimeoutError \\
delete & tableName, recordId & deleteResult & RecordNotFound, DatabaseConnectionError, TimeoutError \\
\bottomrule
\end{tabular}
\end{center}

\subsection{Semantics}

\subsubsection{State Variables}

\begin{itemize}
  \item databaseState: Represents the current state of the database and its connection status
\end{itemize}

\subsubsection{Environment Variables}

\begin{itemize}
  \item databaseServer: represents the external database system holding the data
  \item fileStorageService: represents the external file storage service for storing receipt images/documents
\end{itemize}

\subsubsection{Assumptions}

\begin{itemize}
  \item The databaseServer is accessible and running
  \item The fileStorageService is accessible and running
  \item The input parameters for each access program are valid and conform to the expected formats
  \item The module has the necessary permissions to perform CRUD operations on the database and file storage service
  \item The database schema is predefined and known to the module
  \item Authentication and authorization are handled by other modules before accessing this module
  \item Network connectivity to the databaseServer and fileStorageService is stable during operations
\end{itemize}

\subsubsection{Access Routine Semantics}

\noindent query(queryString, tableName, params):
\begin{itemize}
  \item transition: Verify databaseState is connected. Execute the queryString with the provided parameters against the databaseServer.
  \item output: Return the queryResult containing the results of the executed query.
  \item exception: Raise InvalidQuery if the queryString is malformed, DatabaseConnectionError if unable to connect to the database, or TimeoutError if the operation exceeds defaultTimeout.
\end{itemize}

\noindent insert(tableName, document):
\begin{itemize}
  \item transition: Verify databaseState is connected. Insert the provided document into the specified tableName in the databaseServer.
  \item output: Return the insertResult indicating success and the recordId of the newly created record.
  \item exception: Raise InvalidRecordData if the document is malformed, DatabaseConnectionError if unable to connect to the database, or TimeoutError if the operation exceeds defaultTimeout.  
\end{itemize}

\noindent update(tableName, recordId, updatedFields):
\begin{itemize}
  \item transition: Verify databaseState is connected. Update the record with recordId in tableName using the provided updatedFields.
  \item output: Return the updateResult indicating success and the recordId of the updated record.
  \item exception: Raise RecordNotFound if the recordId does not exist, InvalidUpdateData if updatedFields are malformed, DatabaseConnectionError if unable to connect to the database, or TimeoutError if the operation exceeds defaultTimeout.  
\end{itemize}

\noindent delete(tableName, recordId):
\begin{itemize}
  \item transition: Verify databaseState is connected. Delete the record with recordId from tableName.
  \item output: Return the deleteResult indicating success.
  \item exception: Raise RecordNotFound if the recordId does not exist, DatabaseConnectionError if unable to connect to the database, or TimeoutError if the operation exceeds defaultTimeout. 
\end{itemize}

\newpage

\bibliographystyle {plainnat}
\bibliography {../../../refs/References}

\newpage

\section{Appendix} \label{Appendix}

\wss{Extra information if required}

\newpage{}

\section*{Appendix --- Reflection}

\wss{Not required for CAS 741 projects}

The information in this section will be used to evaluate the team members on the
graduate attribute of Problem Analysis and Design.

\input{../../Reflection.tex}

\begin{enumerate}
  \item What went well while writing this deliverable? 
  \item What pain points did you experience during this deliverable, and how
    did you resolve them?
  \item Which of your design decisions stemmed from speaking to your client(s)
  or a proxy (e.g. your peers, stakeholders, potential users)? For those that
  were not, why, and where did they come from?
  \item While creating the design doc, what parts of your other documents (e.g.
  requirements, hazard analysis, etc), it any, needed to be changed, and why?
  \item What are the limitations of your solution?  Put another way, given
  unlimited resources, what could you do to make the project better? (LO\_ProbSolutions)
  \item Give a brief overview of other design solutions you considered.  What
  are the benefits and tradeoffs of those other designs compared with the chosen
  design?  From all the potential options, why did you select the documented design?
  (LO\_Explores)
\end{enumerate}


\end{document}