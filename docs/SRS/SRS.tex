% THIS DOCUMENT IS TAILORED TO REQUIREMENTS FOR SCIENTIFIC COMPUTING.  IT SHOULDN'T
% BE USED FOR NON-SCIENTIFIC COMPUTING PROJECTS
\documentclass[12pt]{article}

\usepackage{amsmath, mathtools}
\usepackage{amsfonts}
\usepackage{amssymb}
\usepackage{graphicx}
\usepackage{colortbl}
\usepackage{xr}
\usepackage{hyperref}
\usepackage{longtable}
\usepackage{xfrac}
\usepackage{tabularx}
\usepackage{float}
\usepackage{siunitx}
\usepackage{booktabs}
\usepackage{caption}
\usepackage{pdflscape}
\usepackage{afterpage}

\usepackage[round]{natbib}

%\usepackage{refcheck}

\hypersetup{
    bookmarks=true,         % show bookmarks bar?
      colorlinks=true,       % false: boxed links; true: colored links
    linkcolor=red,          % color of internal links (change box color with linkbordercolor)
    citecolor=green,        % color of links to bibliography
    filecolor=magenta,      % color of file links
    urlcolor=cyan           % color of external links
}

\input{../Comments}
%% Common Parts

\newcommand{\progname}{Software Engineering} % PUT YOUR PROGRAM NAME HERE
\newcommand{\authname}{Team \#5, Money Making Mauraders
\\ Zhenia Sigayev
\\ Justin Ho
\\ Thomas Wang
\\ Michael Shi
\\ Johnny Qu
}


\usepackage{hyperref}
    \hypersetup{colorlinks=true, linkcolor=blue, citecolor=blue, filecolor=blue,
                urlcolor=blue, unicode=false}
    \urlstyle{same}
                                


% For easy change of table widths
\newcommand{\colZwidth}{1.0\textwidth}
\newcommand{\colAwidth}{0.13\textwidth}
\newcommand{\colBwidth}{0.82\textwidth}
\newcommand{\colCwidth}{0.1\textwidth}
\newcommand{\colDwidth}{0.05\textwidth}
\newcommand{\colEwidth}{0.8\textwidth}
\newcommand{\colFwidth}{0.17\textwidth}
\newcommand{\colGwidth}{0.5\textwidth}
\newcommand{\colHwidth}{0.28\textwidth}

% Used so that cross-references have a meaningful prefix
\newcounter{defnum} %Definition Number
\newcommand{\dthedefnum}{GD\thedefnum}
\newcommand{\dref}[1]{GD\ref{#1}}
\newcounter{datadefnum} %Datadefinition Number
\newcommand{\ddthedatadefnum}{DD\thedatadefnum}
\newcommand{\ddref}[1]{DD\ref{#1}}
\newcounter{theorynum} %Theory Number
\newcommand{\tthetheorynum}{TM\thetheorynum}
\newcommand{\tref}[1]{TM\ref{#1}}
\newcounter{tablenum} %Table Number
\newcommand{\tbthetablenum}{TB\thetablenum}
\newcommand{\tbref}[1]{TB\ref{#1}}
\newcounter{assumpnum} %Assumption Number
\newcommand{\atheassumpnum}{A\theassumpnum}
\newcommand{\aref}[1]{A\ref{#1}}
\newcounter{goalnum} %Goal Number
\newcommand{\gthegoalnum}{GS\thegoalnum}
\newcommand{\gsref}[1]{GS\ref{#1}}
\newcounter{instnum} %Instance Number
\newcommand{\itheinstnum}{IM\theinstnum}
\newcommand{\iref}[1]{IM\ref{#1}}
\newcounter{reqnum} %Requirement Number
\newcommand{\rthereqnum}{R\thereqnum}
\newcommand{\rref}[1]{R\ref{#1}}
\newcounter{nfrnum} %NFR Number
\newcommand{\rthenfrnum}{NFR\thenfrnum}
\newcommand{\nfrref}[1]{NFR\ref{#1}}
\newcounter{lcnum} %Likely change number
\newcommand{\lthelcnum}{LC\thelcnum}
\newcommand{\lcref}[1]{LC\ref{#1}}

\usepackage{fullpage}

\begin{document}

\title{Software Requirements Specification for \progname: MES Finance Tracking Platform} 
\author{\authname}
\date{\today}
	
\maketitle

~\newpage

\pagenumbering{roman}

\tableofcontents

~\newpage

\section*{Revision History}

\begin{tabularx}{\textwidth}{p{3cm}p{2cm}X}
  \toprule {\bf Date} & {\bf Version} & {\bf Notes}\\
    \midrule
      Date 1 & 1.0 & Notes\\
      Date 2 & 1.1 & Notes\\
  \bottomrule
\end{tabularx}

\newpage
\pagenumbering{arabic}


%-------------------------------------
% 1. Purpose of the Project
%-------------------------------------
\section{Purpose of the Project}

  \subsection{User Business}


  \subsection{Goals of the Project}
  \begin{itemize}
      \item \textbf{Goal 1:} Brief explanation of goal 1. Add more goals.
  \end{itemize}

%-------------------------------------
% 2. System Overview
%-------------------------------------
\section{System Overview}

  \subsection{Architecture Diagram}
    \begin{figure}[H]
        \centering
        \includegraphics[width=0.8\textwidth]{architecture_diagram.png}
        \caption{TODO: ADD CAPTION FOR ARCHITECTURE DIAGRAM. This should be a general overview of the flow of information and how 
        features interact with each other. No technologies etc should be included here.}
        \label{fig:architecture_diagram}
    \end{figure}

  \subsection{Technology Stack Overview}
    \begin{figure}[H]
        \centering
        \includegraphics[width=0.8\textwidth]{tech_stack.png}
        \caption{TODO: ADD CAPTION FOR TECH STACK OVERVIEW. This should be a general overview of the technologies used in the system and how they interact with each other.
        This should show how the frontend and backend interact with each other and any external services used.}
        \label{fig:tech_stack_overview}
    \end{figure}

%-------------------------------------
% 3. Stakeholders
%-------------------------------------
\section{Stakeholders}

  \subsection{McMaster Engineering Society (MES)}
  
  \subsection{Direct Stakeholders}
  
    \subsubsection{MES Club Leadership}
    Describe in more detail and bullet points.

    \subsubsection{Administrative Staff}
    \subsubsection{MES Club Members}
    \subsubsection{MES Student Developers}

  \subsection{Indirect Stakeholders}
    \subsubsection{Faculty Advisors}

  \subsection{Personas}
  \subsection{Priorities Assigned to Stakeholders}
  \subsection{Stakeholder Participation}


%-------------------------------------
% 4. Mandated Constraints
%-------------------------------------
\section{Mandated Constraints}


\subsection{Solution Constraints}

  \subsection{Current MES Platform Overview}
  \subsection{Current MES Finance Tracker}
  \subsection{Cost Constraints}
  \subsection{Off-the-Shelf Software}
  \subsection{Schedule Constraints}
  \subsection{Workflow Constraints}
  \subsection{Enterprise Constraints}


%-------------------------------------
% 5. Terminology, Acronyms, and Technologies
%-------------------------------------
\section{Terminology, Acronyms, and Technologies}

  \subsection{Terminology}
  \subsection{Acronyms}
  \subsection{Technologies}

%-------------------------------------
% 6. Relevant Facts And Assumptions
%-------------------------------------
\section{Relevant Facts And Assumptions}
  \subsection{Relevant Facts}
  \subsection{Business Rules}
  \subsection{Assumptions}

%-------------------------------------
% 7. The Scope of the Work
%-------------------------------------
\section{The Scope of the Work}
  \subsection{The Current Situation}
  \subsection{The Context of the Work}
  \subsection{Work Partitioning}
  \subsection{Specifying a Business Use Case (BUC)}

%-------------------------------------
% 8. Business Data Model and Data Dictionary
%-------------------------------------
\section{Business Data Model and Data Dictionary}
  \subsection{Business Data Model}
  \subsection{Data Dictionary}

%-------------------------------------
% 9. Functional Requirements
%-------------------------------------
\section{Functional Requirements}
  \subsection{Functional Requirements}
    \begin{itemize}
        \item Example
      \end{itemize}

%-------------------------------------
% 10. Non-Functional Requirements
%-------------------------------------
\section{Non-Functional Requirements}
  \subsection{Non-Functional Requirements}
    \begin{itemize}
        \item Example
      \end{itemize}

%-------------------------------------
% 11. Look and Feel Requirements
%-------------------------------------
\section{Look and Feel Requirements}
  \subsection{Appearance Requirements}
    \begin{itemize}
      \item Example
    \end{itemize}

  \subsection{Style Requirements}
    \begin{itemize}
      \item Example
    \end{itemize}

%-------------------------------------
% 12. Usability and Humanity Requirements
%-------------------------------------
\section{Usability and Humanity Requirements}
  \subsection{Ease of Use Requirements}
    \begin{itemize}
      \item Example
    \end{itemize}

  \subsection{Personalization and Internationalization Requirements}
    \begin{itemize}
      \item Example
    \end{itemize}

  \subsection{Learning Requirements}
    \begin{itemize}
      \item Example
    \end{itemize}

  \subsection{Understandability and Politeness Requirements}
    \begin{itemize}
      \item Example
    \end{itemize}

  \subsection{Accessibility Requirements}
    \begin{itemize}
      \item Example
    \end{itemize}

%-------------------------------------
% 13. Performance Requirements
%-------------------------------------
\section{Performance Requirements}
  \subsection{Speed and Latency Requirements}
    \begin{itemize}
      \item Example
    \end{itemize}

  \subsection{Safety-Critical Requirements}
    \begin{itemize}
      \item Example
    \end{itemize}

  \subsection{Precision or Accuracy Requirements}
    \begin{itemize}
      \item Example
    \end{itemize}

  \subsection{Robustness or Fault-Tolerance Requirements}
    \begin{itemize}
      \item Example
    \end{itemize}

  \subsection{Capacity Requirements}
    \begin{itemize}
      \item Example
    \end{itemize}

  \subsection{Scalability or Extensibility Requirements}
    \begin{itemize}
      \item Example
    \end{itemize}

  \subsection{Longevity Requirements}
    \begin{itemize}
      \item Example
    \end{itemize}

%-------------------------------------
% 14. Operational and Environmental Requirements
%-------------------------------------
\section{Operational and Environmental Requirements}
  \subsection{Expected Physical Environment}
  \begin{itemize}
    \item Example
  \end{itemize}

  \subsection{Wider Environment Requirements}
    \begin{itemize}
      \item Example
    \end{itemize}

  \subsection{Requirements for Interfacing with Adjacent Systems}
    \begin{itemize}
      \item Example
    \end{itemize}

  \subsection{Productization Requirements}
    \begin{itemize}
      \item Example
    \end{itemize}

  \subsection{Release Requirements}
    \begin{itemize}
      \item Example
    \end{itemize}

%-------------------------------------
% 15. Maintainability and Support Requirements
%-------------------------------------
\section{Maintainability and Support Requirements}
  \subsection{Maintenance Requirements}
    \begin{itemize}
      \item Example
    \end{itemize}

  \subsection{Supportability Requirements}
    \begin{itemize}
      \item Example
    \end{itemize}

  \subsection{Adaptability Requirements}
    \begin{itemize}
      \item Example
    \end{itemize}

%-------------------------------------
% 16. Security Requirements
%-------------------------------------
\section{Security Requirements}
  \subsection{Access Requirements}
    \begin{itemize}
      \item Example
    \end{itemize}

  \subsection{Integrity Requirements}
    \begin{itemize}
      \item Example
    \end{itemize}

  \subsection{Privacy Requirements}
    \begin{itemize}
      \item Example
    \end{itemize}

  \subsection{Audit Requirements}
    \begin{itemize}
      \item Example
    \end{itemize}

  \subsection{Immunity Requirements}
    \begin{itemize}
      \item Example
    \end{itemize}

%-------------------------------------
% 17. Cultural Requirements
%-------------------------------------
\section{Cultural Requirements}
  \subsection{Cultural Requirements}
    \begin{itemize}
      \item Example
    \end{itemize}

%-------------------------------------
% 18. Compliance Requirements
%-------------------------------------
\section{Compliance Requirements}
  \subsection{Legal Requirements}
    \begin{itemize}
      \item Example
    \end{itemize}

  \subsection{Standards Compliance Requirements}
    \begin{itemize}
      \item Example
    \end{itemize}

%-------------------------------------
% 19. Open Issues
%-------------------------------------
\section{Open Issues}
  \begin{itemize}
    \item Example
  \end{itemize}

%-------------------------------------
% 20. Off-the-Shelf Solutions
%-------------------------------------
\section{Off-the-Shelf Solutions}
  \subsection{Ready-Made Products}
    \begin{itemize}
      \item Example
    \end{itemize}

  \subsection{Reusable Components}
    \begin{itemize}
      \item Example
    \end{itemize}

  \subsection{Products That Can Be Copied}
    \begin{itemize}
      \item Example
    \end{itemize}

%-------------------------------------
% 21. Likely Changes
%-------------------------------------
\section{Likely Changes}    

%-------------------------------------
% 22. Unlikely Changes
%-------------------------------------
\section{Unlikely Changes}    

%-------------------------------------
% 23. Ideas for Solution
%-------------------------------------
\section{Ideas for Solution}

%-------------------------------------
% 24. Requirements Traceability
%-------------------------------------
\section{Requirements Traceability}

The purpose of the traceability matrices is to provide easy references on what
has to be additionally modified if a certain component is changed.  Every time a
component is changed, the items in the column of that component that are marked
with an ``X'' may have to be modified as well.  Table~\ref{Table:trace} shows the
dependencies of theoretical models, general definitions, data definitions, and
instance models with each other. Table~\ref{Table:R_trace} shows the
dependencies of instance models, requirements, and data constraints on each
other. Table~\ref{Table:A_trace} shows the dependencies of theoretical models,
general definitions, data definitions, instance models, and likely changes on
the assumptions.

\plt{You will have to modify these tables for your problem.}

\plt{The traceability matrix is not generally symmetric.  If GD1 uses A1, that
  means that GD1's derivation or presentation requires invocation of A1.  A1
  does not use GD1.  A1 is ``used by'' GD1.}

\plt{The traceability matrix is challenging to maintain manually.  Please do
  your best.  In the future tools (like Drasil) will make this much easier.}

\afterpage{
\begin{landscape}
\begin{table}[h!]
\centering
\begin{tabular}{|c|c|c|c|c|c|c|c|c|c|c|c|c|c|c|c|c|c|c|c|}
\hline
	& \aref{A_OnlyThermalEnergy}& \aref{A_hcoeff}& \aref{A_mixed}& \aref{A_tpcm}& \aref{A_const_density}& \aref{A_const_C}& \aref{A_Newt_coil}& \aref{A_tcoil}& \aref{A_tlcoil}& \aref{A_Newt_pcm}& \aref{A_charge}& \aref{A_InitTemp}& \aref{A_OpRangePCM}& \aref{A_OpRange}& \aref{A_htank}& \aref{A_int_heat}& \aref{A_vpcm}& \aref{A_PCM_state}& \aref{A_Pressure} \\
\hline
\tref{T_COE}        & X& & & & & & & & & & & & & & & & & & \\ \hline
\tref{T_SHE}        & & & & & & & & & & & & & & & & & & & \\ \hline
\tref{T_LHE}        & & & & & & & & & & & & & & & & & & & \\ \hline
\dref{NL}           & & X& & & & & & & & & & & & & & & & & \\ \hline
\dref{ROCT}         & & & X& X& X& X& & & & & & & & & & & & & \\ \hline
\ddref{FluxCoil}    & & & & & & & X& X& X& & & & & & & & & & \\ \hline
\ddref{FluxPCM}     & & & X& X& & & & & & X& & & & & & & & & \\ \hline
\ddref{D_HOF}       & & & & & & & & & & & & & & & & & & & \\ \hline
\ddref{D_MF}        & & & & & & & & & & & & & & & & & & & \\ \hline
\iref{ewat}         & & & & & & & & & & & X& X& & X& X& X& & & X \\ \hline
\iref{epcm}         & & & & & & & & & & & & X& X& & & X& X& X& \\ \hline
\iref{I_HWAT}       & & & & & & & & & & & & & & X& & & & & X \\ \hline
\iref{I_HPCM}       & & & & & & & & & & & & & X& & & & & X & \\ \hline
\lcref{LC_tpcm}     & & & & X& & & & & & & & & & & & & & & \\ \hline
\lcref{LC_tcoil}    & & & & & & & & X& & & & & & & & & & & \\ \hline
\lcref{LC_tlcoil}   & & & & & & & & & X& & & & & & & & & & \\ \hline
\lcref{LC_charge}   & & & & & & & & & & & X& & & & & & & & \\ \hline
\lcref{LC_InitTemp} & & & & & & & & & & & & X& & & & & & & \\ \hline
\lcref{LC_htank}    & & & & & & & & & & & & & & & X& & & & \\
\hline
\end{tabular}
\caption{Traceability Matrix Showing the Connections Between Assumptions and Other Items}
\label{Table:A_trace}
\end{table}
\end{landscape}
}

\begin{table}[h!]
\centering
\begin{tabular}{|c|c|c|c|c|c|c|c|c|c|c|c|c|c|c|c|c|c|c|c|c|c|c|c|}
\hline        
	& \tref{T_COE}& \tref{T_SHE}& \tref{T_LHE}& \dref{NL}& \dref{ROCT} & \ddref{FluxCoil}& \ddref{FluxPCM} & \ddref{D_HOF}& \ddref{D_MF}& \iref{ewat}& \iref{epcm}& \iref{I_HWAT}& \iref{I_HPCM} \\
\hline
\tref{T_COE}     & & & & & & & & & & & & & \\ \hline
\tref{T_SHE}     & & & X& & & & & & & & & & \\ \hline
\tref{T_LHE}     & & & & & & & & & & & & & \\ \hline
\dref{NL}        & & & & & & & & & & & & & \\ \hline
\dref{ROCT}      & X& & & & & & & & & & & & \\ \hline
\ddref{FluxCoil} & & & & X& & & & & & & & & \\ \hline
\ddref{FluxPCM}  & & & & X& & & & & & & & & \\ \hline
\ddref{D_HOF}    & & & & & & & & & & & & & \\ \hline
\ddref{D_MF}     & & & & & & & & X& & & & & \\ \hline
\iref{ewat}      & & & & & X& X& X& & & & X& & \\ \hline
\iref{epcm}      & & & & & X& & X& & X& X& & & X \\ \hline
\iref{I_HWAT}    & & X& & & & & & & & & & & \\ \hline
\iref{I_HPCM}    & & X& X& & & & X& X& X& & X& & \\
\hline
\end{tabular}
\caption{Traceability Matrix Showing the Connections Between Items of Different Sections}
\label{Table:trace}
\end{table}

\begin{table}[h!]
\centering
\begin{tabular}{|c|c|c|c|c|c|c|c|}
\hline
	& \iref{ewat}& \iref{epcm}& \iref{I_HWAT}& \iref{I_HPCM}& \ref{sec_DataConstraints}& \rref{R_RawInputs}& \rref{R_MassInputs} \\
\hline
\iref{ewat}            & & X& & & & X& X \\ \hline
\iref{epcm}            & X& & & X& & X& X \\ \hline
\iref{I_HWAT}          & & & & & & X& X \\ \hline
\iref{I_HPCM}          & & X& & & & X& X \\ \hline
\rref{R_RawInputs}     & & & & & & & \\ \hline
\rref{R_MassInputs}    & & & & & & X& \\ \hline
\rref{R_CheckInputs}   & & & & & X& & \\ \hline
\rref{R_OutputInputs}  & X& X& & & & X& X \\ \hline
\rref{R_TempWater}     & X& & & & & & \\ \hline 
\rref{R_TempPCM}       & & X& & & & & \\ \hline
\rref{R_EnergyWater}   & & & X& & & & \\ \hline
\rref{R_EnergyPCM}     & & & & X& & & \\ \hline
\rref{R_VerifyOutput}  & & & X& X& & & \\ \hline
\rref{R_timeMeltBegin} & & X& & & & & \\ \hline
\rref{R_timeMeltEnd}   & & X& & & & & \\ 
\hline
\end{tabular}
\caption{Traceability Matrix Showing the Connections Between Requirements and Instance Models}
\label{Table:R_trace}
\end{table}

The purpose of the traceability graphs is also to provide easy references on
what has to be additionally modified if a certain component is changed.  The
arrows in the graphs represent dependencies. The component at the tail of an
arrow is depended on by the component at the head of that arrow. Therefore, if a
component is changed, the components that it points to should also be
changed. Figure~\ref{Fig_ATrace} shows the dependencies of theoretical models,
general definitions, data definitions, instance models, likely changes, and
assumptions on each other. Figure~\ref{Fig_RTrace} shows the dependencies of
instance models, requirements, and data constraints on each other.

% \begin{figure}[h!]
% 	\begin{center}
% 		%\rotatebox{-90}
% 		{
% 			\includegraphics[width=\textwidth]{ATrace.png}
% 		}
% 		\caption{\label{Fig_ATrace} Traceability Matrix Showing the Connections Between Items of Different Sections}
% 	\end{center}
% \end{figure}


% \begin{figure}[h!]
% 	\begin{center}
% 		%\rotatebox{-90}
% 		{
% 			\includegraphics[width=0.7\textwidth]{RTrace.png}
% 		}
% 		\caption{\label{Fig_RTrace} Traceability Matrix Showing the Connections Between Requirements, Instance Models, and Data Constraints}
% 	\end{center}
% \end{figure}

\section{Development Plan}

\plt{This section is optional.  It is used to explain the plan for developing
  the software.  In particular, this section gives a list of the order in which
  the requirements will be implemented.  In the context of a course  this is
  where you can indicate which requirements will be implemented as part of the
  course, and which will be ``faked'' as future work.  This section can be
  organized as a prioritized list of requirements, or it could should the
  requirements that will be implemented for ``phase 1'', ``phase 2'', etc.}

\section{Values of Auxiliary Constants}

\plt{Show the values of the symbolic parameters introduced in the report.}

\plt{The definition of the requirements will likely call for SYMBOLIC\_CONSTANTS.
Their values are defined in this section for easy maintenance.}

\plt{The value of FRACTION, for the Maintainability NFR would be given here.}

\newpage

\bibliographystyle {plainnat}
\bibliography {../../refs/References}

\newpage

\section*{Appendix --- Reflection}

  \subsection{What went well while writing this deliverable?}
  \begin{itemize}
    \item Zhenia Sigayev
    \item Justin Ho
    \item Thomas Wang
    \item Michael Shi
    \item Johnny Qu
  \end{itemize}

  \subsection{How many of your requirements were inspired by speaking to your
    client(s) or their proxies (e.g. your peers, stakeholders, potential users)?}
  \begin{itemize}
    \item Zhenia Sigayev
    \item Justin Ho
    \item Thomas Wang
    \item Michael Shi
    \item Johnny Qu
  \end{itemize}

  \subsection{What knowledge and skills will the team collectively need to acquire to
    successfully complete this capstone project? This includes domain specific knowledge or software engineering and/or
    computer science knowledge. Skills may be related to technology, or writing,
    or presentation, or team management, etc. You should look to identify at
    least one item for each team member.}
  \begin{itemize}
    \item Zhenia Sigayev
    \item Justin Ho
    \item Thomas Wang
    \item Michael Shi
    \item Johnny Qu
  \end{itemize}

\section*{Appendix --- References}
    \hypertarget{Ref1}{[1]} Example,” Example Institution, https://example.ca/doc/ (accessed Mon. day, year). \\

\input{../Reflection.tex}
\input{../SRS_Reflection.tex}
\end{document}