\documentclass{article}

\usepackage{float}
\restylefloat{table}

\usepackage{booktabs}

\title{Team Contributions: POC\\\progname}

\author{\authname}

\date{}

\input{../Comments}
%% Common Parts

\newcommand{\progname}{Software Engineering} % PUT YOUR PROGRAM NAME HERE
\newcommand{\authname}{Team \#5, Money Making Mauraders
\\ Zhenia Sigayev
\\ Justin Ho
\\ Thomas Wang
\\ Michael Shi
\\ Johnny Qu
}


\usepackage{hyperref}
    \hypersetup{colorlinks=true, linkcolor=blue, citecolor=blue, filecolor=blue,
                urlcolor=blue, unicode=false}
    \urlstyle{same}
                                


\begin{document}

\maketitle

This document summarizes the contributions of each team member up to the POC
Demo.  The time period of interest is the time between the beginning of the term
and the POC demo.

\section{Demo Plans}

We will be demonstrating the following functionality at the POC demo:
\begin{itemize}
    \item Image upload via web interface
    \item OCR processing of uploaded images
    \item Display of extracted text to user
    \item Storage of images and extracted text in database
\end{itemize}
\wss{What will you be demonstrating}

\section{Team Meeting Attendance}

\wss{For each team member how many team meetings have they attended over the
time period of interest.  This number should be determined from the meeting
issues in the team's repo.  The first entry in the table should be the total
number of team meetings held by the team.}

\begin{table}[H]
\centering
\begin{tabular}{ll}
\toprule
\textbf{Student} & \textbf{Meetings}\\
\midrule
Total & 9\\
Justin & 9\\
Zhenia & 8\\
Name 3 & Num\\
Name 4 & Num\\
Name 5 & Num\\
\bottomrule
\end{tabular}
\end{table}

\wss{If needed, an explanation for the counts can be provided here.}

\section{Supervisor/Stakeholder Meeting Attendance}

\wss{For each team member how many supervisor/stakeholder team meetings have
they attended over the time period of interest.  This number should be determined
from the supervisor meeting issues in the team's repo.  The first entry in the
table should be the total number of supervisor and team meetings held by the
team.  If there is no supervisor, there will usually be meetings with
stakeholders (potential users) that can serve a similar purpose.}

\noindent \textbf{Supervisor's Name: } [fill in this information]

\begin{table}[H]
\centering
\begin{tabular}{ll}
\toprule
\textbf{Student} & \textbf{Meetings}\\
\midrule
Total & 2\\
Justin & 2\\
Zhenia & 2\\
Name 3 & Num\\
Name 4 & Num\\
Name 5 & Num\\
\bottomrule
\end{tabular}
\end{table}

\wss{If needed, an explanation for the counts can be provided here.}

\section{Lecture Attendance}

\wss{For each team member how many lectures have they attended over the time
period of interest.  This number should be determined from the lecture issues in
the team's repo. You can find the number of lectures in the time period of
interest by looking at the
\href{https://calendar.google.com/calendar/u/0/embed?src=rnboqiaki1k2la7rpu3bn0um58@group.calendar.google.com&ctz=America/Toronto}
{Google calendar} for the capstone course.}

\wss{NOTE: There will be approximately 13 lectures between the start of class
and the POC demos}

\begin{table}[H]
\centering
\begin{tabular}{ll}
\toprule
\textbf{Student} & \textbf{Lectures}\\
\midrule
Total & 13\\
Justin & 5\\
Zhenia & 4\\
Name 3 & Num\\
Name 4 & Num\\
Name 5 & Num\\
\bottomrule
\end{tabular}
\end{table}

\wss{If needed, an explanation for the lecture attendance can be provided here.}

\section{TA Document Discussion Attendance}

\wss{For each team member how many of the informal document discussion meetings
with the TA were attended over the time period of interest.}

\noindent \textbf{TA's Name: } [fill in this information]

\begin{table}[H]
\centering
\begin{tabular}{ll}
\toprule
\textbf{Student} & \textbf{Lectures}\\
\midrule
Total & 4\\
Justin & 4\\
Zhenia & 4\\
Name 3 & Num\\
Name 4 & Num\\
Name 5 & Num\\
\bottomrule
\end{tabular}
\end{table}

\wss{If needed, an explanation for the attendance can be provided here.}

\section{Commits}

\wss{For each team member how many commits to the main branch have been made
over the time period of interest.  The total is the total number of commits for
the entire team since the beginning of the term.  The percentage is the
percentage of the total commits made by each team member.}

\begin{table}[H]
\centering
\begin{tabular}{lll}
\toprule
\textbf{Student} & \textbf{Commits} & \textbf{Percent}\\
\midrule
Total & 90 & 100\% \\
Justin & 18 & 20\% \\
Zhenia & 44 & 49\% \\
Name 3 & Num & \% \\
Name 4 & Num & \% \\
Name 5 & Num & \% \\
\bottomrule
\end{tabular}
\end{table}

Zhenia: My 44 of 90 commits (49\%) include a few commits to port documentation repository to GitHub from GitLab.

\wss{If needed, an explanation for the counts can be provided here.  For
instance, if a team member has more commits to unmerged branches, these numbers
can be provided here.  If multiple people contribute to a commit, git allows for
multi-author commits.}

\section{Issue Tracker}

\wss{For each team member how many issues have they authored (including open and
closed issues (O+C)) and how many have they been assigned (only counting closed
issues (C only)) over the time period of interest.}

\begin{table}[H]
\centering
\begin{tabular}{lll}
\toprule
\textbf{Student} & \textbf{Authored (O+C)} & \textbf{Assigned (C only)}\\
\midrule
Justin & 8 & 0 \\
Zhenia & 16 & 0 \\
Name 3 & Num & Num \\
Name 4 & Num & Num \\
Name 5 & Num & Num \\
\bottomrule
\end{tabular}
\end{table}

\wss{If needed, an explanation for the counts can be provided here.}

\section{CICD}

CICD has been used in the following ways in our project:
\begin{itemize}
    \item automatic updates to out pdfs
\end{itemize}

\wss{Say how CICD will be used in your project}

\section{Team Charter Trigger Items}

Our triggers were mainly based around communication, responsibilties, and scheduling.
\begin{itemize}
    \item communication
    \begin{itemize}
        \item use pull request for code changes
        \item discord of quick coordination
        \item email for formal stakeholder communication
        \item use github issues for task management
    \end{itemize}
    \item responsibilities
    \begin{itemize}
        \item each member is responsible for their assigned issues
        \item members are responsible for attending meetings
        \item team member roles (PM, Dev lead, etc.) are responsible are constantly being rotated and shared
    \end{itemize}
    \item scheduling
    \begin{itemize}
        \item issues are assigned with realistic deadlines
        \item deadlines are met
    \end{itemize}
\end{itemize}

Violated triggers
\begin{itemize}
    \item communication
    \begin{itemize}
        \item Github issues were not always used
    \end{itemize}
    \item responsibilities
    \begin{itemize}
        \item N/A
    \end{itemize}
    \item scheduling
    \begin{itemize}
        \item N/A
    \end{itemize}
\end{itemize}

To address the violation of not always using github issues, we will make a
reminder to use github issues for task management in our next team meeting.


\wss{Provide a summary of the quantified triggers identified in the team's
charter.}

\wss{Provide a list of any violations of the triggers.  If the team wishes, the
violations can be summarized on aggregate, instead of naming specific team
members.}

\wss{Provide a plan to address the violations.  This could include revising the
triggers, if they are found to be too weak, strong or ambiguous.}

\section{Additional Productivity Metrics}

\wss{If your team has additional metrics of productivity, please feel free to
add them to this report.}

\end{document}