\documentclass[12pt, titlepage]{article}

\usepackage{amsmath, mathtools}

\usepackage[round]{natbib}
\usepackage{amsfonts}
\usepackage{amssymb}
\usepackage{graphicx}
\usepackage{colortbl}
\usepackage{xr}
\usepackage{hyperref}
\usepackage{longtable}
\usepackage{xfrac}
\usepackage{tabularx}
\usepackage{float}
\usepackage{siunitx}
\usepackage{booktabs}
\usepackage{multirow}
\usepackage[section]{placeins}
\usepackage{caption}
\usepackage{fullpage}

\hypersetup{
bookmarks=true,     % show bookmarks bar?
colorlinks=true,       % false: boxed links; true: colored links
linkcolor=red,          % color of internal links (change box color with linkbordercolor)
citecolor=blue,      % color of links to bibliography
filecolor=magenta,  % color of file links
urlcolor=cyan          % color of external links
}

\usepackage{array}

\externaldocument{../../SRS/SRS}

\input{../../Comments}
%% Common Parts

\newcommand{\progname}{Software Engineering} % PUT YOUR PROGRAM NAME HERE
\newcommand{\authname}{Team \#5, Money Making Mauraders
\\ Zhenia Sigayev
\\ Justin Ho
\\ Thomas Wang
\\ Michael Shi
\\ Johnny Qu
}


\usepackage{hyperref}
    \hypersetup{colorlinks=true, linkcolor=blue, citecolor=blue, filecolor=blue,
                urlcolor=blue, unicode=false}
    \urlstyle{same}
                                


\begin{document}

\title{Module Interface Specification for \progname{}}

\author{\authname}

\date{\today}

\maketitle

\pagenumbering{roman}

\section{Revision History}

\begin{tabularx}{\textwidth}{p{3cm}p{2cm}X}
\toprule {\bf Date} & {\bf Version} & {\bf Notes}\\
\midrule
Monday, November 10, 2025 & 1.0 & Initial Document\\
\bottomrule
\end{tabularx}

~\newpage

\section{Symbols, Abbreviations and Acronyms}

See SRS Documentation at \wss{give url}

\wss{Also add any additional symbols, abbreviations or acronyms}

\newpage

\tableofcontents

\newpage

\pagenumbering{arabic}

\section{Introduction}

The following document details the Module Interface Specifications (MIS) for the MES Finance Tracking Platform, a web-based system that streamlines the submission, approval, and monitoring of reimbursement requests for McMaster Engineering Society clubs. By centralizing digital expense forms, supporting documentation, and reviewer workflows, the platform shortens processing times and improves transparency for students and finance administrators alike.

This document specifies the MIS for each software module in the MES Finance Tracking Platform, outlining how components interact to deliver the functionality described in the accompanying design artifacts.

Complementary documents include the System Requirement Specifications, Module Guide, and complete documentation and implementation can be found at \url{https://github.com/zheniasigayev/MES-Finance-Tracking}.

\section{Notation}

\wss{You should describe your notation.  You can use what is below as
  a starting point.}

The structure of the MIS for modules comes from \citet{HoffmanAndStrooper1995},
with the addition that template modules have been adapted from
\cite{GhezziEtAl2003}.  The mathematical notation comes from Chapter 3 of
\citet{HoffmanAndStrooper1995}.  For instance, the symbol := is used for a
multiple assignment statement and conditional rules follow the form $(c_1
\Rightarrow r_1 | c_2 \Rightarrow r_2 | ... | c_n \Rightarrow r_n )$.

The following table summarizes the primitive data types used by \progname. 

\begin{center}
\renewcommand{\arraystretch}{1.2}
\noindent 
\begin{tabular}{l l p{7.5cm}} 
\toprule 
\textbf{Data Type} & \textbf{Notation} & \textbf{Description}\\ 
\midrule
character & char & a single symbol or digit\\
integer & $\mathbb{Z}$ & a number without a fractional component in (-$\infty$, $\infty$) \\
natural number & $\mathbb{N}$ & a number without a fractional component in [1, $\infty$) \\
real & $\mathbb{R}$ & any number in (-$\infty$, $\infty$)\\
\bottomrule
\end{tabular} 
\end{center}

\noindent
The specification of \progname \ uses some derived data types: sequences, strings, and
tuples. Sequences are lists filled with elements of the same data type. Strings
are sequences of characters. Tuples contain a list of values, potentially of
different types. In addition, \progname \ uses functions, which
are defined by the data types of their inputs and outputs. Local functions are
described by giving their type signature followed by their specification.

\section{Module Decomposition}

The following table is taken directly from the Module Guide document for this project.

\begin{table}[h!]
\centering
\begin{tabular}{p{0.3\textwidth} p{0.6\textwidth}}
\toprule
\textbf{Level 1} & \textbf{Level 2}\\
\midrule

{Hardware-Hiding} & \textbullet~M1: Hardware Hiding Module \\
\midrule

\multirow{7}{0.3\textwidth}{Behaviour-Hiding} & Input Parameters\\
& Output Format\\
& Output Verification\\
& Temperature ODEs\\
& Energy Equations\\ 
& Control Module\\
& Specification Parameters Module\\
\midrule

\multirow{3}{0.3\textwidth}{Software Decision} & {Sequence Data Structure}\\
& ODE Solver\\
& Plotting\\
\bottomrule

\end{tabular}
\caption{Module Hierarchy}
\label{TblMH}
\end{table}

\newpage
\section{MIS of Hardware Hiding Module (M1)}\label{sec:hardware-hiding}

\subsection{Module}
Hardware Hiding Module (M1)

\subsection{Uses}
None.

\subsection{Syntax}

\subsubsection{Exported Constants}
\begin{itemize}
  \item \textbf{RuntimeVersion}: string --- Declares the LTS Node.js runtime version made available by the hosting platform (Vercel/AWS) so that the Behaviour-Hiding and Software Decision modules can align their server-side features and package targets.
  \item \textbf{DefaultRegion}: string --- Identifies the primary cloud deployment region used for the platform's compute, database, and object storage endpoints to satisfy latency expectations from the SRS operational environment.
  \item \textbf{MaxConcurrentRequests}: $\mathbb{N}$ --- Upper bound on simultaneous backend invocations guaranteed by the provider, chosen to satisfy the SRS capacity requirement of supporting at least 500 concurrent users.
\end{itemize}

\subsubsection{Exported Access Programs}

\begin{center}
\begin{tabular}{p{3.5cm} p{4.5cm} p{4.5cm} p{3cm}}
\toprule
\textbf{Name} & \textbf{In} & \textbf{Out} & \textbf{Exceptions} \\
\midrule
getRuntimeContext & - & RuntimeContext & ConfigNotFound \\
provisionPersistence & PersistenceTarget, PersistencePolicy & PersistenceHandle & ProvisioningError \\
openFileChannel & StorageRequest & StorageHandle & StorageError \\
resolveSecret & SecretKey & SecretValue & SecretNotFound \\
\bottomrule
\end{tabular}
\end{center}

\subsection{Semantics}

\subsubsection{State Variables}
\begin{itemize}
  \item \textbf{activeHandles}: sequence of PersistenceHandle --- Tracks open database or object storage connections issued through the module.
  \item \textbf{cachedSecrets}: map SecretKey $\rightarrow$ SecretValue --- Maintains decrypted secrets for the lifetime of a runtime invocation to minimize calls to the provider's secret manager.
\end{itemize}

\subsubsection{Environment Variables}
\begin{itemize}
  \item \textbf{cloudPlatform}: descriptor of the underlying infrastructure (e.g., Vercel Edge Functions or AWS Lambda + S3) that supplies compute, networking, and storage abstractions.
  \item \textbf{processEnv}: map string $\rightarrow$ string exposing environment variables injected by the hosting provider, including connection strings and API keys referenced in the Development Plan.
  \item \textbf{filesystem}: ephemeral file system mount allocated per invocation for staging uploads before they are persisted to long-term storage.
  \item \textbf{networkInterface}: outbound HTTPS client managed by the runtime for communicating with third-party services such as SendGrid or future payment APIs.
\end{itemize}

\subsubsection{Assumptions}
\begin{itemize}
  \item The cloud platform guarantees availability targets and auto-scaling properties outlined in the SRS environment and capacity requirements.
  \item Required environment variables (database URIs, storage bucket identifiers, API credentials) are injected securely by the deployment pipeline defined in the Development Plan.
  \item Serverless function instances share no persistent disk state; any durable data must be written through \texttt{provisionPersistence} or \texttt{openFileChannel}.
\end{itemize}

\subsubsection{Access Routine Semantics}
\noindent\textbf{getRuntimeContext}($\,$)
\begin{itemize}
  \item transition: None.
  \item output: Returns a RuntimeContext record containing \textbf{RuntimeVersion}, \textbf{DefaultRegion}, exposed environment variables, and runtime limits (memory, execution time) advertised by \textbf{cloudPlatform}.
  \item exception: ConfigNotFound is raised if mandatory runtime metadata is missing from \textbf{processEnv}.
\end{itemize}

\noindent\textbf{provisionPersistence}(target, policy)
\begin{itemize}
  \item transition: Adds a PersistenceHandle corresponding to the requested \textit{target} (document store, object storage, or cache) to \textbf{activeHandles}. The handle encapsulates connection pooling or signed URLs as required by the target.
  \item output: Returns the newly created PersistenceHandle configured according to \textit{policy} (e.g., read/write role, retention period).
  \item exception: ProvisioningError if the target infrastructure is unreachable or policy constraints cannot be satisfied by the provider.
\end{itemize}

\noindent\textbf{openFileChannel}(request)
\begin{itemize}
  \item transition: Streams the binary payload described by \textit{request} from \textbf{filesystem} to an object storage location derived from \textbf{cloudPlatform}. Updates \textbf{activeHandles} with the resulting StorageHandle for lifecycle management.
  \item output: Returns a StorageHandle containing the canonical URI and checksum for the persisted object so that the Receipt Processing module can reference it.
  \item exception: StorageError when the payload exceeds provider-imposed limits or when the storage backend reports an error.
\end{itemize}

\noindent\textbf{resolveSecret}(key)
\begin{itemize}
  \item transition: If \textit{key} is not present in \textbf{cachedSecrets}, retrieves the secret value from the provider-managed vault and caches it for the remaining invocation lifespan.
  \item output: Returns the SecretValue associated with \textit{key} so downstream modules can authenticate with external services (e.g., MongoDB, SendGrid).
  \item exception: SecretNotFound when the requested key is absent from the vault or access is denied.
\end{itemize}

\subsubsection{Local Functions}
None.

\subsubsection{Considerations}
\begin{itemize}
  \item This module virtualizes physical infrastructure so higher-level modules can operate independent of Vercel/AWS specific APIs, satisfying the information-hiding intent of the Module Guide.
  \item Capacity thresholds published through \textbf{MaxConcurrentRequests} ensure the Request Handler and Notification modules can plan throttling logic that respects SRS capacity and scalability constraints.
  \item Secrets resolved via \texttt{resolveSecret} enable future integration with the MES monorepo while keeping credential management centralized, as described in the Development Plan.
\end{itemize}

~\newpage

\section{MIS of \wss{Module Name}} \label{Module} \wss{Use labels for
  cross-referencing}

\wss{You can reference SRS labels, such as R\ref{R_Inputs}.}

\wss{It is also possible to use \LaTeX for hypperlinks to external documents.}

\subsection{Module}

\wss{Short name for the module}

\subsection{Uses}


\subsection{Syntax}

\subsubsection{Exported Constants}

\subsubsection{Exported Access Programs}

\begin{center}
\begin{tabular}{p{2cm} p{4cm} p{4cm} p{2cm}}
\hline
\textbf{Name} & \textbf{In} & \textbf{Out} & \textbf{Exceptions} \\
\hline
\wss{accessProg} & - & - & - \\
\hline
\end{tabular}
\end{center}

\subsection{Semantics}

\subsubsection{State Variables}

\wss{Not all modules will have state variables.  State variables give the module
  a memory.}

\subsubsection{Environment Variables}

\wss{This section is not necessary for all modules.  Its purpose is to capture
  when the module has external interaction with the environment, such as for a
  device driver, screen interface, keyboard, file, etc.}

\subsubsection{Assumptions}

\wss{Try to minimize assumptions and anticipate programmer errors via
  exceptions, but for practical purposes assumptions are sometimes appropriate.}

\subsubsection{Access Routine Semantics}

\noindent \wss{accessProg}():
\begin{itemize}
\item transition: \wss{if appropriate} 
\item output: \wss{if appropriate} 
\item exception: \wss{if appropriate} 
\end{itemize}

\wss{A module without environment variables or state variables is unlikely to
  have a state transition.  In this case a state transition can only occur if
  the module is changing the state of another module.}

\wss{Modules rarely have both a transition and an output.  In most cases you
  will have one or the other.}

\subsubsection{Local Functions}

\wss{As appropriate} \wss{These functions are for the purpose of specification.
  They are not necessarily something that is going to be implemented
  explicitly.  Even if they are implemented, they are not exported; they only
  have local scope.}

\newpage

\bibliographystyle {plainnat}
\bibliography {../../../refs/References}

\newpage

\section{Appendix} \label{Appendix}

\wss{Extra information if required}

\newpage{}

\section*{Appendix --- Reflection}

\wss{Not required for CAS 741 projects}

The information in this section will be used to evaluate the team members on the
graduate attribute of Problem Analysis and Design.

\input{../../Reflection.tex}

\begin{enumerate}
  \item What went well while writing this deliverable? 
  \item What pain points did you experience during this deliverable, and how
    did you resolve them?
  \item Which of your design decisions stemmed from speaking to your client(s)
  or a proxy (e.g. your peers, stakeholders, potential users)? For those that
  were not, why, and where did they come from?
  \item While creating the design doc, what parts of your other documents (e.g.
  requirements, hazard analysis, etc), it any, needed to be changed, and why?
  \item What are the limitations of your solution?  Put another way, given
  unlimited resources, what could you do to make the project better? (LO\_ProbSolutions)
  \item Give a brief overview of other design solutions you considered.  What
  are the benefits and tradeoffs of those other designs compared with the chosen
  design?  From all the potential options, why did you select the documented design?
  (LO\_Explores)
\end{enumerate}


\end{document}