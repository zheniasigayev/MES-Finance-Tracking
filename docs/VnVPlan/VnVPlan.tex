\documentclass[12pt, titlepage]{article}

\usepackage{booktabs}
\usepackage{tabularx}
\usepackage{longtable}
\usepackage{hyperref}
\hypersetup{
    colorlinks,
    citecolor=blue,
    filecolor=black,
    linkcolor=red,
    urlcolor=blue
}
\usepackage[round]{natbib}

%% Comments

\usepackage{color}

\newif\ifcomments\commentstrue %displays comments
%\newif\ifcomments\commentsfalse %so that comments do not display

\ifcomments
\newcommand{\authornote}[3]{\textcolor{#1}{[#3 ---#2]}}
\newcommand{\todo}[1]{\textcolor{red}{[TODO: #1]}}
\else
\newcommand{\authornote}[3]{}
\newcommand{\todo}[1]{}
\fi

\newcommand{\wss}[1]{\authornote{magenta}{SS}{#1}} 
\newcommand{\plt}[1]{\authornote{cyan}{TPLT}{#1}} %For explanation of the template
\newcommand{\an}[1]{\authornote{cyan}{Author}{#1}}

%% Common Parts

\newcommand{\progname}{Software Engineering} % PUT YOUR PROGRAM NAME HERE
\newcommand{\authname}{Team \#5, Money Making Mauraders
\\ Zhenia Sigayev
\\ Justin Ho
\\ Thomas Wang
\\ Michael Shi
\\ Johnny Qu
}


\usepackage{hyperref}
    \hypersetup{colorlinks=true, linkcolor=blue, citecolor=blue, filecolor=blue,
                urlcolor=blue, unicode=false}
    \urlstyle{same}
                                


\begin{document}

\title{System Verification and Validation Plan for \progname{}} 
\author{\authname}
\date{\today}
	
\maketitle

\pagenumbering{roman}

\section*{Revision History}

\begin{tabularx}{\textwidth}{p{3cm}p{2cm}X}
\toprule {\bf Date} & {\bf Version} & {\bf Notes}\\
\midrule
Date 1 & 1.0 & Notes\\
Date 2 & 1.1 & Notes\\
\bottomrule
\end{tabularx}

~\\
\wss{The intention of the VnV plan is to increase confidence in the software.
However, this does not mean listing every verification and validation technique
that has ever been devised.  The VnV plan should also be a \textbf{feasible}
plan. Execution of the plan should be possible with the time and team available.
If the full plan cannot be completed during the time available, it can either be
modified to ``fake it'', or a better solution is to add a section describing
what work has been completed and what work is still planned for the future.}

\wss{The VnV plan is typically started after the requirements stage, but before
the design stage.  This means that the sections related to unit testing cannot
initially be completed.  The sections will be filled in after the design stage
is complete.  the final version of the VnV plan should have all sections filled
in.}

\newpage

\tableofcontents

\listoftables
\wss{Remove this section if it isn't needed}

\listoffigures
\wss{Remove this section if it isn't needed}

\newpage

\section{Symbols, Abbreviations, and Acronyms}

\renewcommand{\arraystretch}{1.2}
\begin{tabular}{l l} 
  \toprule		
  \textbf{symbol} & \textbf{description}\\
  \midrule 
  BUC & Business Use Cases\\
  \bottomrule
\end{tabular}\\

\wss{symbols, abbreviations, or acronyms --- you can simply reference the SRS
  \citep{SRS} tables, if appropriate}

\wss{Remove this section if it isn't needed}

\newpage

\pagenumbering{arabic}

This document ... \wss{provide an introductory blurb and roadmap of the
  Verification and Validation plan}

\section{General Information}

\subsection{Summary}

\wss{Say what software is being tested.  Give its name and a brief overview of
  its general functions.}

The MES Finance Tracking Platform is a web application that centralizes club reimbursement submissions, 
reviews, and approvals for the MES with role based access and receipt storage. The front end user interface 
will be tested for usability and functionality to ensure a smooth user experience. The back end data storage
will be tested for data integrity and security to ensure sensitive financial information is protected. The database
will be tested for performance and reliability to ensure quick access to financial records. 

\subsection{Objectives}

\wss{State what is intended to be accomplished.  The objective will be around
  the qualities that are most important for your project.  You might have
  something like: ``build confidence in the software correctness,''
  ``demonstrate adequate usability.'' etc.  You won't list all of the qualities,
  just those that are most important.}

\wss{You should also list the objectives that are out of scope.  You don't have 
the resources to do everything, so what will you be leaving out.  For instance, 
if you are not going to verify the quality of usability, state this.  It is also 
worthwhile to justify why the objectives are left out.}

\wss{The objectives are important because they highlight that you are aware of 
limitations in your resources for verification and validation.  You can't do everything, 
so what are you going to prioritize?  As an example, if your system depends on an 
external library, you can explicitly state that you will assume that external library 
has already been verified by its implementation team.}

\noindent In Scope Objectives:
\begin{itemize}
  \item Demonstrate software correctness of end-to-end user interface reimbursement workflow
  \item Ensure data integrity and security of sensitive financial information 
  \item Meet all performance/capacity targets for database access times
  \item Ensure financial accuracy of reimbursement camera captures and calculations
\end{itemize}

\noindent Out of Scope Objectives:
\begin{itemize}
  \item Comprehensive usability testing for all potential user personas due to limited resources.
\end{itemize}


\subsection{Extras}

\wss{Summarize the extras (if any) that were tackled by this project.  Extras
can include usability testing, code walkthroughs, user documentation, formal
proof, GenderMag personas, Design Thinking, etc.  Extras should have already
been approved by the course instructor as included in your problem statement.
You can use a pull request to update your extras (in TeamComposition.csv or
Repos.csv) if your plan changes as a result of the VnV planning exercise.}

N/A 

\subsection{Relevant Documentation}

\wss{Reference relevant documentation.  This will definitely include your SRS
  and your other project documents (design documents, like MG, MIS, etc).  You
  can include these even before they are written, since by the time the project
  is done, they will be written.  You can create BibTeX entries for your
  documents and within those entries include a hyperlink to the documents.}

% \citet{SRS}
\begin{itemize}
  \item SRS Documentation
    \begin{itemize}
      \item The SRS is the formal documentation of the functional and non-functional requirements for the project.
      All requirements that must be verified and validated are specified and elaborated on in the SRS.
    \end{itemize}
  \item Hazard Analysis Documentation
    \begin{itemize}
      \item Due to the project's nature of processing and storing financial data, many non-functional requirements 
      arise from hazards related to the security and integrity of data handling. Previously discovered hazards can also inform
      caution points during the verification and validation processes.
    \end{itemize}
  \item Problem Statement and Goals Documentation
    \begin{itemize}
      \item Validation of the requirements proposed in the SRS must relate to the overall goals of the project. The \textit{Problem Statement
      and Goals} documentation includes these high-level goals of the project, and is the primary source of information when the client and/or
      supervisors cannot be reached for clarification. 
    \end{itemize}
  \item Development Plan Documentation
    \begin{itemize}
      \item The development process is to achieve the goals and meet the requirements of the project. In the case that any requirements
      are later invalidated, the plans for development must be modified correspondingly. 
    \end{itemize}
  \item Design Documentation
    \begin{itemize}
      \item The software must be designed such that it meets the verification methods outlined in the VnV. For requirements that can be
      unit-tested, specifics will be included in the unit testing documentation. Many qualitative requirements, however, must be manually
      evaluated according to the procedures set forth in this document.
    \end{itemize}
  \item Unit Testing Plan Documentation
    \begin{itemize}
      \item The \textit{Unit Testing Plan} documentation will be required to reference the VnV Plan for outlined procedures for automatic
      tests, where applicable.
    \end{itemize}
\end{itemize}


\wss{Don't just list the other documents.  You should explain why they are relevant and 
how they relate to your VnV efforts.}

\section{Plan}

\wss{Introduce this section.  You can provide a roadmap of the sections to
  come.}

\subsection{Verification and Validation Team}

\noindent
\begin{tabularx}{\textwidth}{>{\raggedright\arraybackslash}p{3.5cm} >{\raggedright\arraybackslash}p{3.5cm} X}
\toprule
\textbf{Name} & \textbf{Role} & \textbf{Responsibilities}\\
\midrule
Thomas, Justin, Zhenia, Michael, Johnny & Member of MES Finance Tracker Development Team & Development, correctness and usability testing.\\
Luke & Capstone Project Supervisor & Outline all requirements and ensure all requirements have been met and considered.\\
Member of MES Finance Team & Stakeholder & Provide feedback on usability and correctness of financial calculations.\\
Member of MES Clubs & Stakeholder & Provide feedback on usability and correctness of reimbursement workflow.\\
\bottomrule
\end{tabularx}

\subsection{SRS Verification}

\wss{List any approaches you intend to use for SRS verification.  This may
  include ad hoc feedback from reviewers, like your classmates (like your
  primary reviewer), or you may plan for something more rigorous/systematic.}

\wss{If you have a supervisor for the project, you shouldn't just say they will
read over the SRS.  You should explain your structured approach to the review.
Will you have a meeting?  What will you present?  What questions will you ask?
Will you give them instructions for a task-based inspection?  Will you use your
issue tracker?}

\wss{Maybe create an SRS checklist?}

\noindent\textbf{Formal SRS Inspection} \\
\noindent\textbf{Individuals Involved:} Team Members, Luke (Supervisor) \\
\noindent\textbf{Inputs:} SRS Document and Checklist of Requirements and BUC \\
\noindent\textbf{Procedure:}
\begin{enumerate}
  \item Pre-read: all reviewers must read, annotate the SRS documentation, and prepare list of any questions and concerns.
  \item Meeting Walkthrough: reviewers meet to discuss if all requirements listed within the SRS are \emph{clear, verifiable, traceable} and have been met.
  They will also discuss if each BUC has been captured completely and accurately.
  \item Log defects, classify (major/minor), assign owners.
  \item Re-review and redo the process; sign-off when exit criteria are met.
\end{enumerate}
\noindent\textbf{End Criteria:} All major defects resolved; each requirement marked \emph{clear, verifiable, traceable}. \\
\noindent\textbf{Outputs:} Logged defects with resolutions; signed-off SRS document verification. \\


\subsection{Design Verification}

\wss{Plans for design verification}

\wss{The review will include reviews by your classmates}

\wss{Create a checklists?}

\noindent\textbf{Task-Based Validation with Stakeholders} \\
\noindent\textbf{Individuals Involved:} Team Members, MES Finance Team, Club Executives \\
\noindent\textbf{Inputs:} SRS Document and Checklist of Requirements and BUC, Each Stakeholder will provide list of desired functionality from the application \\
\noindent\textbf{Procedure:}
\begin{enumerate}
  \item Pre-read: all team members must review stakeholder desired functionality list and list questions or concerns where applicable.
  \item Meeting Walkthrough: reviewers meet to discuss with each individual stakeholder to see if requirements have been met.
  They will also discuss if each BUC has been captured completely, accurately, and is satisfactory compared to their initial list.
  \item Log defects, classify (major/minor), assign owners.
  \item Re-review and redo the process; sign-off when exit criteria are met.
\end{enumerate}
\noindent\textbf{End Criteria:} All stakeholders are satisfied with the apps functionality and each requirement marked \emph{clear, verifiable, traceable}. \\
\noindent\textbf{Outputs:} Logged defects with resolutions; signed-off SRS document verification. 

\noindent\textbf{Checklist to be Used for Design Verification} \\
\begin{itemize}
  \item Pages use consistent layout and design
  \item Buttons and links look clickable, disabled and hover states are differentiated
  \item Forms are easy to understand and fill out, with clear labels, units/currencies, and instructions
  \item Important actions are visually prominent
  \item Empty states explain what to do next
  \item Member can submit reimbursement with all required fields and receipt uploads
  \item File uploads only accept allowable types/sizes; oversized files return error message
  \item Reviewers can open request, see all details, and Approve/Reject with comments 
  \item Notifications are sent/shown at the correct times (upon submission and decision)
  \item Members can see their own requests and current statuses easily 
  \item Amounts always show 2 decimal points and use the correct currency symbol
  \item Dates and times are clearly outlined 
  \item Each role can only access their permitted features, information, and actions
  \item Approved requests are unable to be edited or resubmitted
  \item Sensitive pages require login, and sessions time out after period of inactivity
  \item Admin features are simple and hard to misuse
  \item Backups/restore plan exists and receipts are included
\end{itemize}

\subsection{Verification and Validation Plan Verification}

\wss{The verification and validation plan is an artifact that should also be
verified.  Techniques for this include review and mutation testing.}

\wss{The review will include reviews by your classmates}

\wss{Create a checklists?}

\subsection{Implementation Verification}

\wss{You should at least point to the tests listed in this document and the unit
  testing plan.}

\wss{In this section you would also give any details of any plans for static
  verification of the implementation.  Potential techniques include code
  walkthroughs, code inspection, static analyzers, etc.}

\wss{The final class presentation in CAS 741 could be used as a code
walkthrough.  There is also a possibility of using the final presentation (in
CAS741) for a partial usability survey.}

\subsection{Automated Testing and Verification Tools}

\wss{What tools are you using for automated testing.  Likely a unit testing
  framework and maybe a profiling tool, like ValGrind.  Other possible tools
  include a static analyzer, make, continuous integration tools, test coverage
  tools, etc.  Explain your plans for summarizing code coverage metrics.
  Linters are another important class of tools.  For the programming language
  you select, you should look at the available linters.  There may also be tools
  that verify that coding standards have been respected, like flake9 for
  Python.}

\wss{If you have already done this in the development plan, you can point to
that document.}

\wss{The details of this section will likely evolve as you get closer to the
  implementation.}


To ensure efficiency and consistency in testing, automated testing and verification
tools will be utilized where possible. As the project spans a wide spread of technologies,
languages, and frameworks, multiple domain-specific tools will be required. The following
section outlines the portions of testing where automated testing tools will be employed; 
specific tools may not yet be specified at the current stage of the project, but potentially
applicable tools will be included as examples. 

\begin{itemize}
  \item \textbf{Linting and Style:} Linting and style tools will be used to improve and preserve
  code readability and maintainability.
  \begin{itemize}
    \item \textbf{Tools:} Pylint (Python), ESLint (Javascript)
  \end{itemize}
  \item \textbf{Unit Testing:} Unit testing will be performed for each part of the project. Due 
  to the project's large technical scope, multiple unit testing suits may be required across 
  different languages/frameworks.
  \begin{itemize}
    \item \textbf{Tools:} pytest (Python), Jest/Cypress/Playwright (Javascript), Postman (API)
  \end{itemize}
  \item \textbf{Load Testing:} As the system is expected to be used for many MES-affiliated clubs, 
  load testing on the APIs and database to ensure availability and scalability.
  \begin{itemize}
    \item \textbf{Tools:} JMeter (Java-based), Locust (Python-based)
  \end{itemize} 
  \item \textbf{Test Scripts:} Any testing functionality not covered by existing tools but requires
  automatic processing may be faciliated through custom testing scripts. As an example, this may 
  include scripts that routinely verify database integrity through queries.
\end{itemize}

\subsection{Software Validation}

\wss{If there is any external data that can be used for validation, you should
  point to it here.  If there are no plans for validation, you should state that
  here.}

\wss{You might want to use review sessions with the stakeholder to check that
the requirements document captures the right requirements.  Maybe task based
inspection?}

\wss{For those capstone teams with an external supervisor, the Rev 0 demo should 
be used as an opportunity to validate the requirements.  You should plan on 
demonstrating your project to your supervisor shortly after the scheduled Rev 0 demo.  
The feedback from your supervisor will be very useful for improving your project.}

\wss{For teams without an external supervisor, user testing can serve the same purpose 
as a Rev 0 demo for the supervisor.}

\wss{This section might reference back to the SRS verification section.}

\noindent\textbf{Stakeholder UAT Against Acceptance Criteria} \\
\noindent\textbf{Individuals Involved:} Team Members, MES Finance Team, Club Executives \\
\noindent\textbf{Points of Clarification:}
\begin{enumerate}
  \item Ensure all MES software acceptance criteria have been met and incorporated into the final product.
  \item Ensure application workflow meets the needs of club executives and MES finance team members.
  \item Ensure all performance and capacity targets have been met in terms of concurrent users, API response times, and database access times.
  \item Ensure all financial calculations and camera receipt captures are accurate and reliable.
  \item Ensure application is usable and intuitive for all intended user personas during Rev 0 demo
\end{enumerate}

\section{System Tests}

\wss{There should be text between all headings, even if it is just a roadmap of
the contents of the subsections.}

This section describes the planned system tests for verifying the MES Finance Tracking Platform. 
Each subsection corresponds to one or more functional requirements defined in the SRS.
 The goal is to ensure that all reimbursement-related workflows—submission, review, and status tracking—perform as intended.

\subsection{Tests for Functional Requirements}

\wss{Subsets of the tests may be in related, so this section is divided into
  different areas.  If there are no identifiable subsets for the tests, this
  level of document structure can be removed.}

\wss{Include a blurb here to explain why the subsections below
  cover the requirements.  References to the SRS would be good here.}

The following tests verify that the MES Finance Tracking Platform correctly implements the functional requirements defined in Section 9.1 of the SRS. 
These tests ensure that all features related to reimbursement submission, review, and audit logging behave as expected.

\subsubsection{Area of Testing — Reimbursement Submission}

This area covers \textbf{Functional Requirements FRQ-1} and \textbf{FRQ-4}, ensuring that club members can submit expense claims with attached digital receipts and that data is permanently stored.

\textbf{Title for Test:} Submit Expense Reimbursement Request \\
\textbf{Test-id:} FRQ-1-SUBMIT

\begin{itemize}
    \item \textbf{Control:} Manual versus Automatic \\
    Automatic (performed using UI automation with manual verification)
    \item \textbf{Initial State:} \\
    The user is logged in as a valid club member with an existing club ID. No reimbursement requests are currently active for that user.
    \item \textbf{Input:} \\
    Receipt file (PDF or image $\leq$10 MB), amount = \$100, reimbursement form fields completed.
    \item \textbf{Output (Expected Result):} \\
    The request appears in the user’s dashboard with status = ``Submitted.'' The uploaded receipt is stored and accessible under that request.
    \item \textbf{Test Case Derivation:} \\
    Based on BUC-2 (``Submitting Reimbursement Request'') and FRQ-1/FRQ-4 in the SRS. Confirms form submission, file upload, and database persistence.
    \item \textbf{How Test Will Be Performed:} \\
    Using Cypress or Playwright to automate UI submission, then verifying MongoDB entries and front-end confirmation.
\end{itemize}

\subsubsection{Area of Testing — Reimbursement Review Workflow}

This area verifies \textbf{FRQ-2}, \textbf{FRQ-3}, and \textbf{FRQ-5}, ensuring that MES reviewers can view submissions, approve/reject them, and that actions are recorded in the audit trail.

\textbf{Title for Test:} Review and Approve Reimbursement Request \\
\textbf{Test-id:} FRQ-2-REVIEW

\begin{itemize}
    \item \textbf{Control:} Manual versus Automatic \\
    Semi-automatic (UI actions by MES reviewer, back-end verification automated)
    \item \textbf{Initial State:} \\
    A reimbursement request exists in ``Submitted'' status. MES reviewer is logged in with admin privileges.
    \item \textbf{Input:} \\
    Reviewer selects request $\rightarrow$ clicks ``Approve'' $\rightarrow$ enters approval note.
    \item \textbf{Output (Expected Result):} \\
    Request status = ``Approved.'' Submitter is notified. Audit log records: \{timestamp, reviewerID, action: Approved\}.
    \item \textbf{Test Case Derivation:} \\
    Based on BUC-3 (``Reviewing Reimbursement Request'') and FRQ-2/FRQ-5 in the SRS.
    \item \textbf{How Test Will Be Performed:} \\
    UI interaction test plus backend log verification using database queries to ensure audit trail persistence.
\end{itemize}

\subsubsection{Area of Testing — Reimbursement Status Tracking}

This test verifies \textbf{FRQ-3} and \textbf{FRQ-6}, ensuring that submitters and authorized users can view real-time reimbursement statuses.

\textbf{Title for Test:} View Reimbursement Status \\
\textbf{Test-id:} FRQ-3-STATUS

\begin{itemize}
    \item \textbf{Control:} Manual versus Automatic \\
    Automatic (API and UI check)
    \item \textbf{Initial State:} \\
    Multiple reimbursement requests exist with different statuses (submitted, under review, approved).
    \item \textbf{Input:} \\
    User navigates to the ``My Requests'' page or queries the API endpoint \texttt{/api/reimbursements}.
    \item \textbf{Output (Expected Result):} \\
    The UI displays the correct status for each request. The API returns accurate data matching the database state.
    \item \textbf{Test Case Derivation:} \\
    Derived from FRQ-3 and FRQ-6, validating that access is role-based and statuses are synchronized with the database.
    \item \textbf{How Test Will Be Performed:} \\
    Automated API tests (Postman/pytest) and front-end UI verification to cross-validate consistency.
\end{itemize}

\subsubsection{Area of Testing — Receipt Persistence and Security}

This area validates the data integrity requirement from \textbf{FRQ-4} and \textbf{Security Requirements INT-1, PRI-1}.

\textbf{Title for Test:} Verify Receipt Storage Security \\
\textbf{Test-id:} FRQ-4-STORAGE

\begin{itemize}
    \item \textbf{Control:} Automatic
    \item \textbf{Initial State:} \\
    One reimbursement submission with a receipt attached.
    \item \textbf{Input:} \\
    Query the database and storage bucket for the receipt file.
    \item \textbf{Output (Expected Result):} \\
    Receipt file is encrypted at rest (per SRS Section 16.3), accessible only to authorized users, and retrievable without corruption.
    \item \textbf{Test Case Derivation:} \\
    From FRQ-4 and SRS Section 16 (INT, PRI).
    \item \textbf{How Test Will Be Performed:} \\
    Check encryption metadata and permissions via the cloud storage API (AWS S3 or Vercel storage).
\end{itemize}

\subsubsection{Area of Testing — Audit Logging}

This area ensures \textbf{FRQ-5} is satisfied by verifying audit log entries for all major actions.

\textbf{Title for Test:} Audit Log Verification \\
\textbf{Test-id:} FRQ-5-AUDIT

\begin{itemize}
    \item \textbf{Control:} Automatic
    \item \textbf{Initial State:} \\
    System contains multiple reimbursement records with various user actions.
    \item \textbf{Input:} \\
    Run log export query for ``Approve'' or ``Reject'' events.
    \item \textbf{Output (Expected Result):} \\
    Each event entry includes timestamp, userID, actionType, and status change. Logs persist for at least 3 years per SRS Section 16.4.
    \item \textbf{Test Case Derivation:} \\
    Derived from FRQ-5 and Audit Requirements [AUD-1, AUD-2].
    \item \textbf{How Test Will Be Performed:} \\
    Automated test script queries audit collection and validates schema and retention policies.
\end{itemize}


\subsection{Tests for Nonfunctional Requirements}

The following tests verify that the MES Finance Tracking Platform meets the nonfunctional requirements defined in the SRS. These include performance, usability, security, and maintainability tests to ensure the system is robust, efficient, and user-friendly.

\subsubsection{Area of Testing — Performance and Latency}

This area covers \textbf{Speed and Latency Requirements [SaL]} and partially overlaps with \textbf{Robustness or Fault-Tolerance Requirements [FLT]}.

\textbf{Title for Test:} API and Page Load Performance Test \\
\textbf{Test-id:} NFL-SAL-1

\begin{itemize}
    \item \textbf{Control:} Automatic
    \item \textbf{Initial State:} \\
    System deployed to staging environment with seeded database (100 clubs, 1,000 reimbursement requests).
    \item \textbf{Input:} \\
    Execute 100 concurrent API requests for reimbursement data and record response times. Load homepage and dashboard in Chrome using Lighthouse.
    \item \textbf{Output (Expected Result):} \\
    API response times average below 1 second; page load times under 3 seconds; file uploads ($<10$ MB) complete within 5 seconds.
    \item \textbf{Test Case Derivation:} \\
    Derived from SRS Section 13.1 [SaL-1, SaL-2, SaL-3].
    \item \textbf{How Test Will Be Performed:} \\
    Automated load testing with tools such as JMeter or Locust, combined with Lighthouse performance scoring. Results summarized in a table comparing observed times to SRS thresholds.
\end{itemize}

\subsubsection{Area of Testing — Usability and Accessibility}

This area validates \textbf{Ease of Use and Learning [EUL]}, \textbf{Accessibility [ABL]}, and \textbf{Style [STY]} requirements.

\textbf{Title for Test:} Usability and Accessibility Evaluation \\
\textbf{Test-id:} NFL-USAB-1

\begin{itemize}
    \item \textbf{Control:} Manual (survey-based)
    \item \textbf{Initial State:} \\
    A functioning system prototype accessible by 5 test users representing different roles (club member, executive, MES reviewer).
    \item \textbf{Input:} \\
    Users perform typical tasks (submit reimbursement, review request, check status). Afterward, they complete a usability survey using a 5-point Likert scale.
    \item \textbf{Output (Expected Result):} \\
    Average ease-of-use rating $\geq$ 4/5. All UI components are keyboard-navigable and meet WCAG 2.1 AA color contrast.
    \item \textbf{Test Case Derivation:} \\
    Based on SRS Sections 11 and 12, especially [EUL-1] and [ABL-1 – ABL-5].
    \item \textbf{How Test Will Be Performed:} \\
    Observation-based usability session followed by survey analysis. Accessibility verified with automated tools (axe-core, Lighthouse accessibility audits).
\end{itemize}

\subsubsection{Area of Testing — Security and Privacy}

This area ensures compliance with \textbf{Access [ACS]}, \textbf{Integrity [INT]}, and \textbf{Privacy [PRI]} requirements.

\textbf{Title for Test:} Role-Based Access and Data Protection Test \\
\textbf{Test-id:} NFL-SEC-1

\begin{itemize}
    \item \textbf{Control:} Automatic with manual verification
    \item \textbf{Initial State:} \\
    Test users include a general member, club executive, and MES admin, each authenticated with proper roles.
    \item \textbf{Input:} \\
    Attempt unauthorized access to restricted endpoints (e.g., reviewing reimbursements as a non-admin). Submit tampered form payloads.
    \item \textbf{Output (Expected Result):} \\
    Unauthorized requests receive HTTP 403 responses; input validation prevents injection/XSS; all communication uses HTTPS (TLS 1.2+). Sensitive fields are encrypted in the database.
    \item \textbf{Test Case Derivation:} \\
    Derived from SRS Section 16: [ACS-1, INT-1, PRI-1].
    \item \textbf{How Test Will Be Performed:} \\
    Automated penetration testing with OWASP ZAP or Burp Suite; verification of encryption and RBAC through API inspection and code review.
\end{itemize}

\subsubsection{Area of Testing — Maintainability and Supportability}

This area validates \textbf{Maintenance [MNT]} and \textbf{Supportability [SUP]} requirements.

\textbf{Title for Test:} Code Quality and Documentation Compliance \\
\textbf{Test-id:} NFL-MNT-1

\begin{itemize}
    \item \textbf{Control:} Static test (manual review and CI enforcement)
    \item \textbf{Initial State:} \\
    Project repository set up with ESLint and Prettier configuration as specified in the SRS.
    \item \textbf{Input:} \\
    Run code through linting and formatting checks; review developer documentation for completeness.
    \item \textbf{Output (Expected Result):} \\
    No critical linting errors; documentation covers setup, deployment, and troubleshooting. All dependencies are updated to current minor versions.
    \item \textbf{Test Case Derivation:} \\
    Derived from SRS Section 15 [MNT-1, SUP-1, SUP-2].
    \item \textbf{How Test Will Be Performed:} \\
    CI pipeline enforces linting and formatting. Manual inspection ensures documentation matches environment setup and maintenance requirements.
\end{itemize}


\subsection{Traceability Between Test Cases and Requirements}

\wss{Provide a table that shows which test cases are supporting which
  requirements.}

  This table maps the high-level Functional Requirements to the supporting Non-Functional Requirements that must be considered during testing to ensure the system's overall quality, security, and performance.

\vspace{0.5cm}

\begin{longtable}{p{0.4\linewidth} p{0.55\linewidth}}
\caption{Traceability Matrix for MES Finance Tracking Platform} \\
\toprule
\textbf{Functional Requirement (FRQ)} & \textbf{Supporting Non-Functional Requirement IDs (NFR)} \\
\midrule
\endfirsthead
\toprule
\textbf{Functional Requirement (FRQ)} & \textbf{Supporting Non-Functional Requirement IDs (NFR)} \\
\midrule
\endhead
\bottomrule
\endfoot
\textbf{FRQ-1:} The system shall allow MES clubs and teams to submit expense claims. &
EUL-1, UaP-2, ABL-1, ABL-3, ABL-5, SaL-1, SaL-2, FLT-2, CAP-1, ACS-1, ACS-2, INT-1, PRI-1, PRI-2, AUD-1, CUL-1, SEI-2, AUL-1, STD-1, STD-2 \\
\midrule
\textbf{FRQ-2:} The system shall provide MES reviewers with tools to efficiently review, approve, or reject the reimbursement requests. &
APP-1, APP-2, STY-3, EUL-1, ABL-1, ABL-5, SaL-1, PRE-1, FLT-1, CAP-1, ACS-1, ACS-2, INT-1, PRI-1, PRI-2, AUD-1, AUD-2, CUL-1, SEI-2, AUL-1, AUL-2, STD-1, STD-2 \\
\midrule
\textbf{FRQ-3:} The system shall track the status of each expense claim (e.g., submitted, under review, approved, rejected, reimbursed). &
APP-2, STY-3, ABL-1, SaL-1, SaL-2, FLT-3, CAP-2, EXT-2, LNG-2, ACS-1, ACS-2, PRI-1, AUD-1, AUD-2, SEI-2, AUL-2, STD-1 \\
\midrule
\textbf{FRQ-4:} The system shall permanently store and retain digital receipt submissions. &
SaL-3, PRE-1, FLT-2, CAP-2, LNG-1, PRI-1, PRI-2, AUD-2, SEI-1, SEI-2, SEI-3, AUL-2, RDL-1, STD-1 \\
\midrule
\textbf{FRQ-5:} The system shall maintain an audit trail that records who submitted, reviewed, approved, or denied each expense claim. &
FLT-2, CAP-2, LNG-1, ACS-2, PRI-2, AUD-1, AUD-2, IMM-1, SEI-2, SEI-3, AUL-1, AUL-2, RDL-1, STD-1, STD-3 \\
\midrule
\textbf{FRQ-6:} The system shall enable access to club expense submissions to the submitters, and other club members with a role greater or equal to that of the submitter, or MES administrators and approvers. &
APP-3, ABL-1, ABL-5, SaL-1, SaL-2, CAP-1, EXT-2, ACS-1, ACS-2, INT-1, PRI-1, PRI-2, AUD-1, SEI-2, AUL-1, AUL-2, STD-1, STD-2 \\
\bottomrule
\end{longtable}

\subsection*{Legend: Non-Functional Requirement Categories}
\begin{itemize}
    \item \textbf{APP / STY:} Look and Feel Requirements (Appearance, Style)
    \item \textbf{EUL / UaP / ABL:} Usability and Humanity Requirements (Ease of Use, Understandability, Accessibility)
    \item \textbf{SaL / PRE / FLT / CAP / EXT / LNG:} Performance Requirements (Speed, Precision, Fault-Tolerance, Capacity, Extensibility, Longevity)
    \item \textbf{ACS / INT / PRI / AUD / IMM:} Security Requirements (Access, Integrity, Privacy, Audit, Immunity)
    \item \textbf{CUL:} Cultural Requirements
    \item \textbf{SEI / AUL / RDL / STD:} Compliance Requirements (Storage/Encryption/Infrastructure, Access/Audit/Logging, Retention/Disposal, Standards)
\end{itemize}

\noindent This matrix ensures that for every function the system performs, the corresponding quality attributes are verified through specific test cases.

\section{Unit Test Description}

\wss{This section should not be filled in until after the MIS (detailed design
  document) has been completed.}

\wss{Reference your MIS (detailed design document) and explain your overall
philosophy for test case selection.}  

\wss{To save space and time, it may be an option to provide less detail in this section.  
For the unit tests you can potentially layout your testing strategy here.  That is, you 
can explain how tests will be selected for each module.  For instance, your test building 
approach could be test cases for each access program, including one test for normal behaviour 
and as many tests as needed for edge cases.  Rather than create the details of the input 
and output here, you could point to the unit testing code.  For this to work, you code 
needs to be well-documented, with meaningful names for all of the tests.}

\subsection{Unit Testing Scope}

\wss{What modules are outside of the scope.  If there are modules that are
  developed by someone else, then you would say here if you aren't planning on
  verifying them.  There may also be modules that are part of your software, but
  have a lower priority for verification than others.  If this is the case,
  explain your rationale for the ranking of module importance.}

\subsection{Tests for Functional Requirements}

\wss{Most of the verification will be through automated unit testing.  If
  appropriate specific modules can be verified by a non-testing based
  technique.  That can also be documented in this section.}

\subsubsection{Module 1}

\wss{Include a blurb here to explain why the subsections below cover the module.
  References to the MIS would be good.  You will want tests from a black box
  perspective and from a white box perspective.  Explain to the reader how the
  tests were selected.}

\begin{enumerate}

\item{test-id1\\}

Type: \wss{Functional, Dynamic, Manual, Automatic, Static etc. Most will
  be automatic}
					
Initial State: 
					
Input: 
					
Output: \wss{The expected result for the given inputs}

Test Case Derivation: \wss{Justify the expected value given in the Output field}

How test will be performed: 
					
\item{test-id2\\}

Type: \wss{Functional, Dynamic, Manual, Automatic, Static etc. Most will
  be automatic}
					
Initial State: 
					
Input: 
					
Output: \wss{The expected result for the given inputs}

Test Case Derivation: \wss{Justify the expected value given in the Output field}

How test will be performed: 

\item{...\\}
    
\end{enumerate}

\subsubsection{Module 2}

...

\subsection{Tests for Nonfunctional Requirements}

\wss{If there is a module that needs to be independently assessed for
  performance, those test cases can go here.  In some projects, planning for
  nonfunctional tests of units will not be that relevant.}

\wss{These tests may involve collecting performance data from previously
  mentioned functional tests.}

\subsubsection{Module ?}
		
\begin{enumerate}

\item{test-id1\\}

Type: \wss{Functional, Dynamic, Manual, Automatic, Static etc. Most will
  be automatic}
					
Initial State: 
					
Input/Condition: 
					
Output/Result: 
					
How test will be performed: 
					
\item{test-id2\\}

Type: Functional, Dynamic, Manual, Static etc.
					
Initial State: 
					
Input: 
					
Output: 
					
How test will be performed: 

\end{enumerate}

\subsubsection{Module ?}

...

\subsection{Traceability Between Test Cases and Modules}

\wss{Provide evidence that all of the modules have been considered.}
				
\bibliographystyle{plainnat}

\bibliography{../../refs/References}

\newpage

\section{Appendix}

This is where you can place additional information.

\subsection{Symbolic Parameters}

The definition of the test cases will call for SYMBOLIC\_CONSTANTS.
Their values are defined in this section for easy maintenance.

\subsection{Usability Survey Questions?}
  \textbf{Ease of Use}
  \begin{enumerate}
      \item On a scale of 1--5, how easy was it to submit a reimbursement request with all required information and receipt uploads?
      \item How intuitive did you find the navigation between different sections of the platform (dashboard, submission form, request status)?
      \item Did you encounter any confusing terminology or unclear instructions while using the system?
  \end{enumerate}

  \textbf{Efficiency}
  \begin{enumerate}
      \item Approximately how long did it take you to complete your first reimbursement submission?
      \item Did the system provide adequate feedback during file uploads and form submission processes?
      \item Were you able to quickly locate and check the status of your previous reimbursement requests?
  \end{enumerate}

  \textbf{Role-Specific Functionality}
  \begin{itemize}
      \item \textbf{For Reviewers:} How easy was it to view all relevant details needed to approve or reject a reimbursement request?
      \item \textbf{For Club Members:} Did you feel confident that your submission was successful after completing the form?
      \item \textbf{For Admins:} How efficiently could you access and manage reimbursement data across multiple clubs?
  \end{itemize}

  \textbf{Accessibility and Design}
  \begin{enumerate}
      \item Were all buttons, links, and interactive elements clearly identifiable and easy to click/tap?
      \item Did the color contrast and text sizing make the interface comfortable to read?
      \item Were you able to complete all tasks using only your keyboard (without a mouse)?
  \end{enumerate}

  \textbf{Error Handling}
  \begin{enumerate}
      \item If you encountered any errors, were the error messages clear and helpful in resolving the issue?
      \item Did the system prevent you from making mistakes (e.g., uploading invalid file types, missing required fields)?
      \item How satisfied were you with the system's response when attempting actions you didn't have permission for?
  \end{enumerate}

  \textbf{Overall Satisfaction}
  \begin{enumerate}
      \item On a scale of 1--5, how likely are you to recommend this platform to other MES clubs?
      \item What feature or aspect of the platform did you find most helpful or well-designed?
      \item What improvements would you suggest to make the platform easier to use?
  \end{enumerate}

\newpage{}
\section*{Appendix --- Reflection}

\wss{This section is not required for CAS 741}

The information in this section will be used to evaluate the team members on the
graduate attribute of Lifelong Learning.

The purpose of reflection questions is to give you a chance to assess your own
learning and that of your group as a whole, and to find ways to improve in the
future. Reflection is an important part of the learning process.  Reflection is
also an essential component of a successful software development process.  

Reflections are most interesting and useful when they're honest, even if the
stories they tell are imperfect. You will be marked based on your depth of
thought and analysis, and not based on the content of the reflections
themselves. Thus, for full marks we encourage you to answer openly and honestly
and to avoid simply writing ``what you think the evaluator wants to hear.''

Please answer the following questions.  Some questions can be answered on the
team level, but where appropriate, each team member should write their own
response:


\begin{enumerate}
  \item What went well while writing this deliverable? 
  \item What pain points did you experience during this deliverable, and how
    did you resolve them?
  \item What knowledge and skills will the team collectively need to acquire to
  successfully complete the verification and validation of your project?
  Examples of possible knowledge and skills include dynamic testing knowledge,
  static testing knowledge, specific tool usage, Valgrind etc.  You should look to
  identify at least one item for each team member.
  \item For each of the knowledge areas and skills identified in the previous
  question, what are at least two approaches to acquiring the knowledge or
  mastering the skill?  Of the identified approaches, which will each team
  member pursue, and why did they make this choice?
\end{enumerate}


\noindent\textbf{Thomas's Reflection:}
\begin{enumerate}
  \item Something that went well for this deliverable was the way that we were all able to split up the work.
  We each individually assigned ourselves a section to write but all came together to discuss the overall structure
  and flow of the document. We were able to complete the document on time and with good quality.
  \item A pain point that was experienced in this deliverable was ensuring that we were able to complete the deliverable
  in a timely manner while balancing other course work. We weren't really able to solve it for this week due to 
  the number of midterms and assignments from other courses, but we will try to plan better for future deliverables.
  \item One knowledge area that I will need to acquire to successfully complete the verification and validation is 
  how to perform dynamic testing for the front end UI framework and how to measure and report database access and API 
  response times. For me personally, I believe this is a skill that will be vital in completing the testing portion of 
  the project but it will also be a useful skill to have in my future career as a software engineer.
  \item For both knowledge areas, there is a structured method and a practical method of learning. The structured method
  involves using online resources, taking courses, or reading documentation to learn the concepts and theory behind the 
  topics. Meanwhile, the practical method involves hands-on experience by actually implementing tests and measuring performance
  on our project. The structured method can be used by all individuals within the group to build a foundation of understanding
  for how to perform the tests, while the practical method will be used by me to actually implement the tests and measure performance
  on our project. I believe that the practical method will be more effective for me as I learn best by trying and learning
  from my mistakes.
\end{enumerate}

\noindent\textbf{Justin's Reflection:}
\begin{enumerate}
  \item For this deliverable, something that went well was the we took initiative to take on work instead of handing out work deliberatly.
  This shows that we are all motivated to complete the deliverable and that we trust each other pick up work to share the workload. Its better than 
  how we previously handled work delegation because theres more autonomy and less micromanagement.
  \item A pain point was aligning out understanding of the deliverable structure and content. Even though we are relatively on the same page, specifics about the 
  requirements and content were inconsistent amongst group members. This caused isseus when deciding how we wanted to create test cases and what should be included. The main resolution
  involved having more discussions and clarifying our understanding of the deliverable requirements.
  \item To properly complete this project, we needed to learn different possible methods to perform dynamic testing for the front end UI framework and how to measure and report database access and API 
  response times. For me, there was a lack of knowledge on different technologies that would allow me to create automated test cases, which is essential for the testing portion of the project.
  \item To acquire the knowledge and skills needed, I want to take more time to find all the possible tools and technologies that can be applied to a situation, compared to just picking from the ones I already know.
  I can also try to learn from others who have more experience in testing and ask for their recommendations on what tools to use. I will most likely pursue the second option because I believe learning from others is more efficient and effective.
\end{enumerate}

\noindent\textbf{Zhenia's Reflection:}
\begin{enumerate}
  \item What went well was our team's ability to self-organize and take ownership of different sections without extensive coordination. This autonomous approach was more efficient than our previous deliverables and allowed us to complete the work on time.
  \item The main challenge was balancing specificity in test cases with our current level of implementation knowledge. I addressed this by focusing on clear inputs and expected outputs while maintaining flexibility for later refinement.
  \item I need to develop proficiency in end-to-end testing for our Next.js application, particularly testing workflows that span the frontend, API, and database layers.
  \item I'll pursue hands-on experimentation with Cypress or Playwright on our existing codebase, as I learn best through practical application rather than theoretical study.
\end{enumerate}

\end{document}