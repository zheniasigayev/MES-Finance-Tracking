\documentclass{article}

\usepackage{tabularx}
\usepackage{booktabs}

\title{Problem Statement and Goals\\\progname}

\author{\authname}

\date{}

%% Comments

\usepackage{color}

\newif\ifcomments\commentstrue %displays comments
%\newif\ifcomments\commentsfalse %so that comments do not display

\ifcomments
\newcommand{\authornote}[3]{\textcolor{#1}{[#3 ---#2]}}
\newcommand{\todo}[1]{\textcolor{red}{[TODO: #1]}}
\else
\newcommand{\authornote}[3]{}
\newcommand{\todo}[1]{}
\fi

\newcommand{\wss}[1]{\authornote{magenta}{SS}{#1}} 
\newcommand{\plt}[1]{\authornote{cyan}{TPLT}{#1}} %For explanation of the template
\newcommand{\an}[1]{\authornote{cyan}{Author}{#1}}

%% Common Parts

\newcommand{\progname}{Software Engineering} % PUT YOUR PROGRAM NAME HERE
\newcommand{\authname}{Team \#5, Money Making Mauraders
\\ Zhenia Sigayev
\\ Justin Ho
\\ Thomas Wang
\\ Michael Shi
\\ Johnny Qu
}


\usepackage{hyperref}
    \hypersetup{colorlinks=true, linkcolor=blue, citecolor=blue, filecolor=blue,
                urlcolor=blue, unicode=false}
    \urlstyle{same}
                                


\begin{document}

\maketitle

\begin{table}[hp]
\caption{Revision History} \label{TblRevisionHistory}
\begin{tabularx}{\textwidth}{llX}
\toprule
\textbf{Date} & \textbf{Developer(s)} & \textbf{Change}\\
\midrule
Date1 & Name(s) & Description of changes\\
Date2 & Name(s) & Description of changes\\
... & ... & ...\\
\bottomrule
\end{tabularx}
\end{table}

\section{Problem Statement}

The engineering department currently uses spreadsheets and manual processes to track club finances. This approach
often results in errors from lost reciepts, duplicated claims, and large amount of claims at the end of the year. A centralized
financial tracking software system is needed to streamline and automate this process.

The software will take clear inputs such as budgets, expenses, transactions, receipts, and any other financial
information. The outputs will include reports, summaries, and visual dashboards. This system will be kept simple
so it is easy to use and maintain for future student groups.

Stakeholders include student group leadership teams who manage budget, administrative staff who monitor spending and 
issue reimbursements, and faculty advisors. The environment the software will run on is the department's existing digital 
infrastructure with the software being in the form of a web platform

The importance of the problem lies in improving financial tracking efficiency, reducing the reporting time
and support better financial decision making for student activities. The goals of the project are measurable: minimize
errors in transaction logging and reciept submissions, encourage continuous submissions over the school year rather than 
a large quantity at the end of the year, and provide clear visualization that makes budgets and spending easy to interpret.


\wss{You should check your problem statement with the
\href{https://github.com/smiths/capTemplate/blob/main/docs/Checklists/ProbState-Checklist.pdf}
{problem statement checklist}.} 

\wss{You can change the section headings, as long as you include the required
information.}

\subsection{Problem}
\begin{itemize}
    \item MES reimbursment process is manual and slow to complete
    \item expense submissions can come from multiple places making them difficult to access
    \item Hard to keep track of payments and receipts
  \end{itemize}

\subsection{Inputs and Outputs}
Inputs: 
\begin{itemize}
  \item Relevant user information (i.e. name)
  \item Reimbursement amount
  \item Proof of expense
    \begin{itemize}
      \item Picture of physical receipt
      \item Digital copy of receipt (pdf, email, etc)
    \end{itemize}
\end{itemize}
Outputs:
\begin{itemize}
  \item Visual dashboard to navigate features 
    \begin{itemize}
      \item Reimbursement request history
      \item Status of reimbursement requests
    \end{itemize}
  \item Financial reports/summaries (for transaction details)
  \item Confirmations when reimbursement request status changes (status examples: submitted, pending, approved, rejected)
\end{itemize}


\wss{Characterize the problem in terms of 'high level' inputs and outputs.  
Use abstraction so that you can avoid details.}

\subsection{Stakeholders}
\begin{itemize}
  \item Group leadership teams in charge of managing MES budget
  \item Administrative staff in charge of monitoring spending provide reimbursements
  \item Faculty advisors who determine what expenses are eligible for reimbursement
\end{itemize}

\subsection{Environment}

\wss{Hardware and Software Environment}
Software and Technology Stack:
\begin{itemize}
  \item \textbf{Languages:}
    \begin{itemize}
      \item TypeScript (primary)
      \item CSS
    \end{itemize}
  \item \textbf{Runtime:}
    \begin{itemize}
      \item Node.js
    \end{itemize}
  \item \textbf{Database:}
    \begin{itemize}
      \item MongoDB Community Server
    \end{itemize}
  \item \textbf{Core Frameworks:}
    \begin{itemize}
      \item Next.js (React-based framework for server-side rendering, routing, static site generation)
      \item React (UI library)
      \item Tailwind CSS (utility-first CSS framework)
      \item NextUI (UI component library)
    \end{itemize}
  \item \textbf{Major Libraries:}
    \begin{itemize}
      \item @mui/material (Material UI for React)
      \item @tanstack/react-query (Data fetching, state management)
      \item axios (HTTP client)
      \item next-auth (authentication)
    \end{itemize}
  \item \textbf{Other Configuration/Plugin Files:}
    \begin{itemize}
      \item next-sitemap (sitemap generation)
      \item pnpm/yarn (dependency management)
    \end{itemize}
\end{itemize}

\section{Goals}

\section{Stretch Goals}

\section{Extras}

\wss{For CAS 741: State whether the project is a research project. This
designation, with the approval (or request) of the instructor, can be modified
over the course of the term.}

\wss{For SE Capstone: List your extras.  Potential extras include usability
testing, code walkthroughs, user documentation, formal proof, GenderMag
personas, Design Thinking, etc.  (The full list is on the course outline and in
Lecture 02.) Normally the number of extras will be two.  Approval of the extras
will be part of the discussion with the instructor for approving the project.
The extras, with the approval (or request) of the instructor, can be modified
over the course of the term.}

\newpage{}

\section*{Appendix --- Reflection}

\wss{Not required for CAS 741}

The purpose of reflection questions is to give you a chance to assess your own
learning and that of your group as a whole, and to find ways to improve in the
future. Reflection is an important part of the learning process.  Reflection is
also an essential component of a successful software development process.  

Reflections are most interesting and useful when they're honest, even if the
stories they tell are imperfect. You will be marked based on your depth of
thought and analysis, and not based on the content of the reflections
themselves. Thus, for full marks we encourage you to answer openly and honestly
and to avoid simply writing ``what you think the evaluator wants to hear.''

Please answer the following questions.  Some questions can be answered on the
team level, but where appropriate, each team member should write their own
response:


\begin{enumerate}
    \item What went well while writing this deliverable? 
    \item What pain points did you experience during this deliverable, and how
    did you resolve them?
    \item How did you and your team adjust the scope of your goals to ensure
    they are suitable for a Capstone project (not overly ambitious but also of
    appropriate complexity for a senior design project)?
\end{enumerate}  

\end{document}